This manuscript
(\href{https://vsmjuetz.github.io/manubot_drmed/v/511028e13bfc7760f0d9efb648daf8ab85a6ba6d/}{permalink})
was automatically generated
from \href{https://github.com/vsmjuetz/manubot_drmed/tree/511028e13bfc7760f0d9efb648daf8ab85a6ba6d}{vsmjuetz/manubot\_drmed@511028e}
on July 8, 2025.

Published: April 19, 2025

\hypertarget{authors}{%
\subsection{Authors}\label{authors}}

\leavevmode\vadjust pre{\hypertarget{correspondence}{}}%
✉ --- Correspondence possible via \href{https://github.com/vsmjuetz/manubot_drmed/issues}{GitHub Issues}

\hypertarget{abschnitt-3-kurzfassung_abstract}{%
\section{Abschnitt 3: Kurzfassung\_Abstract}\label{abschnitt-3-kurzfassung_abstract}}

\hypertarget{kurzfassung}{%
\subsection{\texorpdfstring{\textbf{\emph{Kurzfassung}}}{Kurzfassung}}\label{kurzfassung}}

\textbf{Motivation.} Patienten mit lokal fortgeschrittenen Kopf-Hals-Plattenepithelkarzinomen erhalten standardmäßig eine primäre oder postoperative Radiochemotherapie. Dabei zeigen Patienten mit der gleichen Tumorentität und analogen Staging und Grading nicht selten einen sehr unterschiedlichen klinischen Verlauf bezüglich des Überlebens, Metastasierung und Ansprechen auf die Standardtherapie. Deshalb ist es wichtig, geeignete klinische Parameter und biologische Eigenschaften des Tumorgewebes zur besseren Prognoseschätzung und individuellen Therapieoptimierung zu finden (\href{about:blank\#_ENREF_9}{Baumann \& Krause, 2010}; \href{about:blank\#_ENREF_52}{Lohaus et al., 2014}). Dieses Projekt ist Teil einer multizentrischen Studie der Radioonkologie-Gruppe des Deutschen Konsortiums für Translationale Krebsforschung (DKTK-ROG).

\textbf{Fragestellung.} In diesem Projekt soll untersucht werden, inwieweit CD44 als potenzieller Tumorstammzellmarker in Patienten mit lokal fortgeschrittenem Kopf-Hals-Plattenepithelkarzinomen nach postoperativer Radiochemotherapie eine prognostische Rolle spielt. Für die weitere Stratifizierung soll darüber hinaus die CD44-Expression getrennt für die Patientenkollektive mit HPV16 DNA-positiven und -negativen Tumoren analysiert werden.

\textbf{Methodik.} An allen acht DKTK-Standorten wurden insgesamt 221 Patienten mit Mundhöhlen-, Oropharynx- und Hypopharynxkarzinomen in diese retrospektive Studie eingeschlossen (\href{about:blank\#_ENREF_52}{Lohaus et al., 2014}). Alle Patienten haben im Zeitraum von 2005 bis 2010 eine postoperative Cisplatin-haltige Radiochemotherapie erhalten. In FFPE-Material von 195 Primärtumoren wurde die Proteinexpression von CD44 durch immunhistochemische Färbung an Tissue-Micro-Arrays untersucht.

\textbf{Ergebnis.} Um die prognostische Relevanz von CD44 für lokal fortgeschrittene Kopf-Hals-Plattenepithelkarzinome bei postoperativer Radiochemotherapie zu überprüfen, wurde in dieser Arbeit die CD44-Proteinexpression im prätherapeutischen Tumorgewebematerial mit verschiedenen klinischen Endpunkten korreliert. Univariate Analysen zeigten eine signifikante Assoziation der CD44-Proteinexpression mit der loko-regionären Tumorkontrolle (p = 0,008), aber nicht mit den sekundären Endpunkten Fernmetastasen-freies Überleben (p = 0,075) oder Gesamtüberleben (p = 0,089). Ähnliche Effekte zeigten univariate Analysen der Subgruppe der HPV16 DNA-negativen Kopf-Hals-Plattenepithelkarzinomen. Patienten dieser Subgruppe zeigten einen statistischen Trend für eine bessere loko-regionäre Tumorkontrolle im Vergleich zu Patienten mit HPV16 DNA-negativen und CD44-positiven Tumoren (p = 0,05). Die CD44-Assoziation mit den sekundären Endpunkten in dieser Subgruppe war nicht signifikant.

\textbf{Schlussfolgerung.} Wir konnten zeigen, dass eine Überexpression von CD44 in lokal fortgeschrittenen Kopf-Hals-Plattenepithelkarzinomen mit einer schlechten loko-regionären Kontrolle nach postoperativer Radiochemotherapie assoziiert ist. Zudem stellen Tumorstammzellen neben dem HPV-Status der Tumoren möglicherweise einen weiteren Stratifizierungsparameter zur Individualisierung der postoperativen Radiochemotherapie in Patienten mit Kopf-Hals-Plattenepithelkarzinomen dar. Zur weiteren Evaluierung der klinischen Anwendbarkeit sollten die gewonnenen Ergebnisse in prospektiven Validierungsstudien unter standardisierten Bedingungen weiter geprüft werden.

\hypertarget{abstract}{%
\subsection{\texorpdfstring{\textbf{\emph{Abstract}}}{Abstract}}\label{abstract}}

\textbf{Motivation.} Patients with locally advanced head and neck squamous cell carcinoma receive primary or postoperative chemoradiotherapy as standard. Patients with the same tumor entity and analogous staging and grading often show very different clinical courses in terms of survival, metastasis, and response to standard therapy. It is therefore important to find suitable clinical parameters and biological properties of the tumor tissue for a better prognosis and individual therapy optimization (Baumann \& Krause, 2010). This project is part of a multicenter study by the radiation oncology group of the German Consortium for Translational Cancer Research (DKTK-ROG).

\textbf{Objective.} The aim of this project is to investigate the extent to which CD44 plays a prognostic role as a potential tumor stem cell marker in patients with locally advanced head and neck squamous cell carcinoma after postoperative chemoradiotherapy. For further stratification, the CD44 expression is to be analyzed separately for the patient population with HPV16 DNA-positive and -negative tumors.

\textbf{Methodology.} At all eight DKTK locations, a total of 221 patients with oral cavity, oropharyngeal, and hypopharyngeal carcinoma were included in this retrospective study (Lohaus et al., 2014). All patients received postoperative cisplatin-containing chemoradiotherapy between 2005 and 2010. The protein expression of CD44 in FFPE material from 195 primary tumors was examined by immunohistochemical staining on tissue microarrays.

\textbf{Result.} To test the prognostic relevance of CD44 for locally advanced head and neck squamous cell carcinoma in postoperative chemoradiotherapy, the CD44 protein expression in the pretreatment tumor tissue material was correlated with various clinical endpoints. Univariate analyses showed a significant association of CD44 protein expression with loco-regional tumor control (p = 0.008), but not with the secondary endpoints distant metastasis-free survival (p = 0.075) or overall survival (p = 0.089). Univariate analyses of the subgroup of HPV16 DNA-negative head and neck squamous cell carcinomas showed similar effects. Patients in this subgroup showed a statistical trend for better loco-regional tumor control compared to patients with HPV16 DNA-negative and CD44-positive tumors (p = 0.05). The CD44 association with the secondary endpoints in this subgroup was not significant.

\textbf{Conclusion.} We were able to show that overexpression of CD44 in locally advanced head and neck squamous cell carcinoma is associated with poor loco-regional control after postoperative chemoradiotherapy. In addition to the HPV status of the tumors, tumor stem cells may represent a further stratification parameter for the individualization of postoperative chemoradiotherapy in patients with head and neck squamous cell carcinoma. To further evaluate the clinical applicability, the results obtained should be further checked in prospective validation studies under standardized conditions.

\hypertarget{abstract-1}{%
\subsection{Abstract}\label{abstract-1}}

\hypertarget{abschnitt-2-inhaltsverzeichnis}{%
\section{Abschnitt 2: Inhaltsverzeichnis}\label{abschnitt-2-inhaltsverzeichnis}}

\begin{itemize}
\tightlist
\item
  \protect\hyperlink{kurzfassung}{\textbf{Kurzfassung}}
\item
  \protect\hyperlink{abstract}{\textbf{Abstract}}
\item
  \protect\hyperlink{1-hintergrund}{1 Hintergrund}

  \begin{itemize}
  \tightlist
  \item
    \protect\hyperlink{1-1-kopf-hals-plattenepithelkarzinom}{1.1 Kopf-Hals-Plattenepithelkarzinom}

    \begin{itemize}
    \tightlist
    \item
      \protect\hyperlink{1-1-1-epidemiologie}{1.1.1 Epidemiologie}
    \item
      \protect\hyperlink{1-1-2-uxe4tiologie-und-risikofaktoren}{1.1.2 Ätiologie und Risikofaktoren}
    \item
      \protect\hyperlink{1-1-3-pathogenese}{1.1.3 Pathogenese}
    \item
      \protect\hyperlink{1-1-4-anatomie-histopathologie-und-tumorklassifikation}{1.1.4 Anatomie, Histopathologie und Tumorklassifikation}
    \item
      \protect\hyperlink{1-1-5-klassifikation-und-stadien-nach-ajcc}{1.1.5 Klassifikation und Stadien nach AJCC}
    \item
      \protect\hyperlink{1-1-6-diagnostik-staging}{1.1.6 Diagnostik \& Staging}
    \item
      \protect\hyperlink{1-1-7-therapie}{1.1.7 Therapie}
    \item
      \protect\hyperlink{1-1-8-prognose}{1.1.8 Prognose}
    \end{itemize}
  \item
    \protect\hyperlink{1-2-tumorstammzellen-tumorstammzelltheorie}{1.2 Tumorstammzellen \& Tumorstammzelltheorie}
  \item
    \protect\hyperlink{1-3-tumorstammzellen-als-biomarker-fuxfcr-die-individualisierung-in-der-strahlentherapie}{1.3 Tumorstammzellen als Biomarker für die Individualisierung in der Strahlentherapie}
  \item
    \protect\hyperlink{1-4-forschungsstand-und-forschungsluxfccken}{1.4 Forschungsstand und Forschungslücken}

    \begin{itemize}
    \tightlist
    \item
      \protect\hyperlink{1-4-1-aktuelle-erkenntnisse-zur-rolle-von-cd44-in-hnscc}{1.4.1 Aktuelle Erkenntnisse zur Rolle von CD44 in HNSCC}
    \item
      \protect\hyperlink{1-4-2-vorhandene-studien-zur-cd44-expression-und-klinischen-endpunkten-bei-hnscc}{1.4.2 Vorhandene Studien zur CD44-Expression und klinischen Endpunkten bei HNSCC}
    \item
      \protect\hyperlink{1-4-3-forschungsluxfccken}{1.4.3 Forschungslücken}
    \end{itemize}
  \end{itemize}
\item
  \protect\hyperlink{2-fragestellunghypothese}{2 Fragestellung/Hypothese}
\item
  \protect\hyperlink{3-material-und-methoden}{3 Material und Methoden}

  \begin{itemize}
  \tightlist
  \item
    \protect\hyperlink{3-1-studiendesign}{3.1 Studiendesign}
  \item
    \protect\hyperlink{3-2-patientenkollektiv-und-tumormaterial}{3.2 Patientenkollektiv und Tumormaterial}
  \item
    \protect\hyperlink{3-3-immunhistologische-analysen}{3.3 Immunhistologische Analysen}

    \begin{itemize}
    \tightlist
    \item
      \protect\hyperlink{3-3-1-tissue-microarray}{3.3.1 Tissue-Microarray}
    \item
      \protect\hyperlink{3-3-2-immunhistologische-reaktionen}{3.3.2 Immunhistologische Reaktionen}
    \item
      \protect\hyperlink{3-3-3-immunhistologische-auswertung}{3.3.3 Immunhistologische Auswertung}
    \end{itemize}
  \item
    \protect\hyperlink{3-4-statistische-methoden-und-klinischen-endpunkten}{3.4 Statistische Methoden und klinische Endpunkte}
  \end{itemize}
\item
  \protect\hyperlink{4-ergebnisse}{4 Ergebnisse}

  \begin{itemize}
  \tightlist
  \item
    \protect\hyperlink{4-1-deskriptive-analyse-der-studienpopulation}{4.1 Deskriptive Analyse der Studienpopulation}

    \begin{itemize}
    \tightlist
    \item
      \protect\hyperlink{4-1-1-patientenmerkmale-und-tumorcharakteristik}{4.1.1 Patientenmerkmale und Tumorcharakteristik}
    \item
      \protect\hyperlink{4-1-2-behandlungsmerkmale}{4.1.2 Behandlungsmerkmale}
    \end{itemize}
  \item
    \protect\hyperlink{4-2-ereigniszeitanalysen}{4.2 Ereigniszeitanalysen}

    \begin{itemize}
    \tightlist
    \item
      \protect\hyperlink{4-2-1-deskription-der-ereigniszeitdaten-im-beobachtungszeitraum}{4.2.1 Deskription der Ereigniszeitdaten im Beobachtungszeitraum}
    \item
      \protect\hyperlink{4-2-2-univariate-analysen-zur-prognostischen-relevanz-von-cd44}{4.2.2 Univariate Analysen zur prognostischen Relevanz von CD44}
    \item
      \protect\hyperlink{4-2-3-univariate-analysen-zur-identifizierung-weiterer-prognostischer-parameter}{4.2.3 Univariate Analysen zur Identifizierung weiterer prognostischer Parameter}
    \item
      \protect\hyperlink{4-2-4-multivariate-analysen}{4.2.4 Multivariate Analysen}
    \end{itemize}
  \end{itemize}
\item
  \protect\hyperlink{5-diskussion}{5 Diskussion}

  \begin{itemize}
  \tightlist
  \item
    \protect\hyperlink{5-1-diskussion-der-methoden}{5.1 Diskussion der Methoden}

    \begin{itemize}
    \tightlist
    \item
      \protect\hyperlink{5-1-1-diskussion-des-studiendesigns}{5.1.1 Diskussion des Studiendesigns}
    \item
      \protect\hyperlink{5-1-2-tissue-microarrays}{5.1.2 Tissue Microarrays}
    \item
      \protect\hyperlink{5-1-3-immunhistochemie}{5.1.3 Immunhistochemie}
    \end{itemize}
  \item
    \protect\hyperlink{5-2-diskussion-der-ergebnisse}{5.2 Diskussion der Ergebnisse}

    \begin{itemize}
    \item
      \protect\hyperlink{5-2-1-einfluss-von-cd44-auf-das-klinische-outcome-von-lokal}{5.2.1 Einfluss von CD44 auf das klinische Outcome von lokal}
    \item
      \protect\hyperlink{5-2-2-hpv-status-als-weiterer-stratifizierungsparameter}{5.2.2 HPV-Status als weiterer Stratifizierungsparameter}
    \item
      \protect\hyperlink{5-2-3-tumorstammzellen-als-biomarker-fuxfcr-die-individualisierung-in-der-strahlentherapie}{5.2.3 Tumorstammzellen als Biomarker für die Individualisierung in der Strahlentherapie}
    \item
      \protect\hyperlink{5-2-4-limitationen-der-studie}{5.2.4 Limitationen der Studie}
    \end{itemize}
  \end{itemize}
\end{itemize}

This manuscript is a template (aka ``rootstock'') for \href{https://manubot.org/}{Manubot}, a tool for writing scholarly manuscripts.
Use this template as a starting point for your manuscript.

The rest of this document is a full list of formatting elements/features supported by Manubot.
Compare the input (\texttt{.md} files in the \texttt{/content} directory) to the output you see below.

\hypertarget{basic-formatting}{%
\subsection{Basic formatting}\label{basic-formatting}}

\textbf{Bold} \textbf{text}

{Semi-bold text}

{Centered text}

{Right-aligned text}

\emph{Italic} \emph{text}

Combined \emph{italics and \textbf{bold}}

\sout{Strikethrough}

\begin{enumerate}
\def\labelenumi{\arabic{enumi}.}
\tightlist
\item
  Ordered list item
\item
  Ordered list item

  \begin{enumerate}
  \def\labelenumii{\alph{enumii}.}
  \tightlist
  \item
    Sub-item
  \item
    Sub-item

    \begin{enumerate}
    \def\labelenumiii{\roman{enumiii}.}
    \tightlist
    \item
      Sub-sub-item
    \end{enumerate}
  \end{enumerate}
\item
  Ordered list item

  \begin{enumerate}
  \def\labelenumii{\alph{enumii}.}
  \tightlist
  \item
    Sub-item
  \end{enumerate}
\end{enumerate}

\begin{itemize}
\tightlist
\item
  List item
\item
  List item
\item
  List item
\end{itemize}

subscript: H\textsubscript{2}O is a liquid

superscript: 2\textsuperscript{10} is 1024.

\href{https://www.google.com/search?q=superscript+generator}{unicode superscripts}⁰¹²³⁴⁵⁶⁷⁸⁹

\href{https://www.google.com/search?q=superscript+generator}{unicode subscripts}₀₁₂₃₄₅₆₇₈₉

A long paragraph of text.
Lorem ipsum dolor sit amet, consectetur adipiscing elit, sed do eiusmod tempor incididunt ut labore et dolore magna aliqua.
Ut enim ad minim veniam, quis nostrud exercitation ullamco laboris nisi ut aliquip ex ea commodo consequat.
Duis aute irure dolor in reprehenderit in voluptate velit esse cillum dolore eu fugiat nulla pariatur.
Excepteur sint occaecat cupidatat non proident, sunt in culpa qui officia deserunt mollit anim id est laborum.

Putting each sentence on its own line has numerous benefits with regard to \href{https://asciidoctor.org/docs/asciidoc-recommended-practices/\#one-sentence-per-line}{editing} and \href{https://rhodesmill.org/brandon/2012/one-sentence-per-line/}{version control}.

Line break without starting a new paragraph by putting\\
two spaces at end of line.

\hypertarget{document-organization}{%
\subsection{Document organization}\label{document-organization}}

Document section headings:

\hypertarget{heading-1}{%
\section{Heading 1}\label{heading-1}}

\hypertarget{heading-2}{%
\subsection{Heading 2}\label{heading-2}}

\hypertarget{heading-3}{%
\subsubsection{Heading 3}\label{heading-3}}

\hypertarget{heading-4}{%
\paragraph{Heading 4}\label{heading-4}}

\hypertarget{heading-5}{%
\subparagraph{Heading 5}\label{heading-5}}

Heading 6

\hypertarget{a-heading-centered-on-its-own-printed-page}{%
\subsubsection{A heading centered on its own printed page}\label{a-heading-centered-on-its-own-printed-page}}

Horizontal rule:

\begin{center}\rule{0.5\linewidth}{0.5pt}\end{center}

\texttt{Heading\ 1}'s are recommended to be reserved for the title of the manuscript.

\texttt{Heading\ 2}'s are recommended for broad sections such as \emph{Abstract}, \emph{Methods}, \emph{Conclusion}, etc.

\texttt{Heading\ 3}'s and \texttt{Heading\ 4}'s are recommended for sub-sections.

\hypertarget{links}{%
\subsection{Links}\label{links}}

Bare URL link: \url{https://manubot.org}

\href{https://manubot.org}{Long link with lots of words and stuff and junk and bleep and blah and stuff and other stuff and more stuff yeah}

\href{https://manubot.org}{Link with text}

\href{https://manubot.org}{Link with hover text}

\href{https://manubot.org}{Link by reference}

\hypertarget{citations}{%
\subsection{Citations}\label{citations}}

Citation by DOI {[}\protect\hyperlink{ref-IhliSZDo}{1}{]}.

Citation by PubMed Central ID {[}\protect\hyperlink{ref-mSMVRkoc}{2}{]}.

Citation by PubMed ID {[}\protect\hyperlink{ref-126Wi5Us4}{3}{]}.

Citation by Wikidata ID {[}\protect\hyperlink{ref-QhC8yJ7V}{4}{]}.

Citation by ISBN {[}\protect\hyperlink{ref-zBPP9YKu}{5}{]}.

Citation by URL {[}\protect\hyperlink{ref-1GGGHdsew}{6}{]}.

Citation by alias {[}\protect\hyperlink{ref-PZMP42Ak}{7}{]}.

Multiple citations can be put inside the same set of brackets {[}\protect\hyperlink{ref-IhliSZDo}{1},\protect\hyperlink{ref-zBPP9YKu}{5},\protect\hyperlink{ref-PZMP42Ak}{7}{]}.
Manubot plugins provide easier, more convenient visualization of and navigation between citations {[}\protect\hyperlink{ref-mSMVRkoc}{2},\protect\hyperlink{ref-126Wi5Us4}{3},\protect\hyperlink{ref-PZMP42Ak}{7},\protect\hyperlink{ref-YuJbg3zO}{8}{]}.

Citation tags (i.e.~aliases) can be defined in their own paragraphs using Markdown's reference link syntax:

\hypertarget{referencing-figures-tables-equations}{%
\subsection{Referencing figures, tables, equations}\label{referencing-figures-tables-equations}}

Figure \ref{fig:square-image}

Figure \ref{fig:wide-image}

Figure \ref{fig:tall-image}

Figure \ref{fig:vector-image}

Table \ref{tbl:bowling-scores}

Equation \ref{eq:regular-equation}

Equation \ref{eq:long-equation}

\hypertarget{quotes-and-code}{%
\subsection{Quotes and code}\label{quotes-and-code}}

\begin{quote}
Quoted text
\end{quote}

\begin{quote}
Quoted block of text

Two roads diverged in a wood, and I---\\
I took the one less traveled by,\\
And that has made all the difference.
\end{quote}

Code \texttt{in\ the\ middle} of normal text, aka \texttt{inline\ code}.

Code block with Python syntax highlighting:

\begin{Shaded}
\begin{Highlighting}[]
\ImportTok{from}\NormalTok{ manubot.cite.doi }\ImportTok{import}\NormalTok{ expand\_short\_doi}

\KeywordTok{def}\NormalTok{ test\_expand\_short\_doi():}
\NormalTok{    doi }\OperatorTok{=}\NormalTok{ expand\_short\_doi(}\StringTok{"10/c3bp"}\NormalTok{)}
    \CommentTok{\# a string too long to fit within page:}
    \ControlFlowTok{assert}\NormalTok{ doi }\OperatorTok{==} \StringTok{"10.25313/2524{-}2695{-}2018{-}3{-}vliyanie{-}enhansera{-}copia{-}i{-}insulyatora{-}gypsy{-}na{-}sintez{-}ernk{-}modifikatsii{-}hromatina{-}i{-}svyazyvanie{-}insulyatornyh{-}belkov{-}vtransfetsirovannyh{-}geneticheskih{-}konstruktsiyah"}
\end{Highlighting}
\end{Shaded}

Code block with no syntax highlighting:

\begin{verbatim}
Exporting HTML manuscript
Exporting DOCX manuscript
Exporting PDF manuscript
\end{verbatim}

\hypertarget{figures}{%
\subsection{Figures}\label{figures}}

\begin{figure}
\hypertarget{fig:square-image}{%
\centering
\includegraphics{https://github.com/manubot/resources/raw/15493970f8882fce22bef829619d3fb37a613ba5/test/square.png}
\caption{\textbf{A square image at actual size and with a bottom caption.}
Loaded from the latest version of image on GitHub.}\label{fig:square-image}
}
\end{figure}

\begin{figure}
\hypertarget{fig:wide-image}{%
\centering
\includegraphics{https://github.com/manubot/resources/raw/15493970f8882fce22bef829619d3fb37a613ba5/test/wide.png}
\caption{\textbf{An image too wide to fit within page at full size.}
Loaded from a specific (hashed) version of the image on GitHub.}\label{fig:wide-image}
}
\end{figure}

\begin{figure}
\hypertarget{fig:tall-image}{%
\centering
\includegraphics[width=\textwidth,height=3in]{https://github.com/manubot/resources/raw/15493970f8882fce22bef829619d3fb37a613ba5/test/tall.png}
\caption{\textbf{A tall image with a specified height.}
Loaded from a specific (hashed) version of the image on GitHub.}\label{fig:tall-image}
}
\end{figure}

\begin{figure}
\hypertarget{fig:vector-image}{%
\centering
\includegraphics[width=\textwidth,height=2.5in]{https://raw.githubusercontent.com/manubot/resources/main/test/vector.svg?sanitize=true}
\caption{\textbf{A vector \texttt{.svg} image loaded from GitHub.}
The parameter \texttt{sanitize=true} is necessary to properly load SVGs hosted via GitHub URLs.
White background specified to serve as a backdrop for transparent sections of the image.
Note that if you want to export to Word (\texttt{.docx}), you need to download the image and reference it locally (e.g.~\texttt{content/images/vector.svg}) instead of using a URL.}\label{fig:vector-image}
}
\end{figure}

\hypertarget{tables}{%
\subsection{Tables}\label{tables}}

\begin{longtable}[]{@{}
  >{\raggedright\arraybackslash}p{(\columnwidth - 8\tabcolsep) * \real{0.2308}}
  >{\centering\arraybackslash}p{(\columnwidth - 8\tabcolsep) * \real{0.1923}}
  >{\centering\arraybackslash}p{(\columnwidth - 8\tabcolsep) * \real{0.1923}}
  >{\centering\arraybackslash}p{(\columnwidth - 8\tabcolsep) * \real{0.1923}}
  >{\centering\arraybackslash}p{(\columnwidth - 8\tabcolsep) * \real{0.1923}}@{}}
\caption{A table with a top caption and specified relative column widths.
\label{tbl:bowling-scores}}\label{tbl:bowling-scores}\tabularnewline
\toprule()
\begin{minipage}[b]{\linewidth}\raggedright
\emph{Bowling Scores}
\end{minipage} & \begin{minipage}[b]{\linewidth}\centering
Jane
\end{minipage} & \begin{minipage}[b]{\linewidth}\centering
John
\end{minipage} & \begin{minipage}[b]{\linewidth}\centering
Alice
\end{minipage} & \begin{minipage}[b]{\linewidth}\centering
Bob
\end{minipage} \\
\midrule()
\endfirsthead
\toprule()
\begin{minipage}[b]{\linewidth}\raggedright
\emph{Bowling Scores}
\end{minipage} & \begin{minipage}[b]{\linewidth}\centering
Jane
\end{minipage} & \begin{minipage}[b]{\linewidth}\centering
John
\end{minipage} & \begin{minipage}[b]{\linewidth}\centering
Alice
\end{minipage} & \begin{minipage}[b]{\linewidth}\centering
Bob
\end{minipage} \\
\midrule()
\endhead
Game 1 & 150 & 187 & 210 & 105 \\
Game 2 & 98 & 202 & 197 & 102 \\
Game 3 & 123 & 180 & 238 & 134 \\
\bottomrule()
\end{longtable}

\begin{longtable}[]{@{}
  >{\raggedright\arraybackslash}p{(\columnwidth - 8\tabcolsep) * \real{0.0511}}
  >{\raggedright\arraybackslash}p{(\columnwidth - 8\tabcolsep) * \real{0.2045}}
  >{\raggedright\arraybackslash}p{(\columnwidth - 8\tabcolsep) * \real{0.1989}}
  >{\raggedright\arraybackslash}p{(\columnwidth - 8\tabcolsep) * \real{0.1989}}
  >{\raggedright\arraybackslash}p{(\columnwidth - 8\tabcolsep) * \real{0.3466}}@{}}
\caption{A table too wide to fit within page.
\label{tbl:constant-digits}}\label{tbl:constant-digits}\tabularnewline
\toprule()
\begin{minipage}[b]{\linewidth}\raggedright
\end{minipage} & \begin{minipage}[b]{\linewidth}\raggedright
Digits 1-33
\end{minipage} & \begin{minipage}[b]{\linewidth}\raggedright
Digits 34-66
\end{minipage} & \begin{minipage}[b]{\linewidth}\raggedright
Digits 67-99
\end{minipage} & \begin{minipage}[b]{\linewidth}\raggedright
Ref.
\end{minipage} \\
\midrule()
\endfirsthead
\toprule()
\begin{minipage}[b]{\linewidth}\raggedright
\end{minipage} & \begin{minipage}[b]{\linewidth}\raggedright
Digits 1-33
\end{minipage} & \begin{minipage}[b]{\linewidth}\raggedright
Digits 34-66
\end{minipage} & \begin{minipage}[b]{\linewidth}\raggedright
Digits 67-99
\end{minipage} & \begin{minipage}[b]{\linewidth}\raggedright
Ref.
\end{minipage} \\
\midrule()
\endhead
pi & 3.14159265358979323846264338327950 & 288419716939937510582097494459230 & 781640628620899862803482534211706 & \href{https://www.piday.org/million/}{\texttt{piday.org}} \\
e & 2.71828182845904523536028747135266 & 249775724709369995957496696762772 & 407663035354759457138217852516642 & \href{https://apod.nasa.gov/htmltest/gifcity/e.2mil}{\texttt{nasa.gov}} \\
\bottomrule()
\end{longtable}

\begin{longtable}[]{@{}ccc@{}}
\caption{A table with merged cells using the \texttt{attributes} plugin.
}\label{tbl:}\tabularnewline
\toprule()
& \textbf{Colors} & \\
\midrule()
\endfirsthead
\toprule()
& \textbf{Colors} & \\
\midrule()
\endhead
\textbf{Size} & \textbf{Text Color} & \textbf{Background Color} \\
big & blue & orange \\
small & black & white \\
\bottomrule()
\end{longtable}

\hypertarget{equations}{%
\subsection{Equations}\label{equations}}

A LaTeX equation:

\begin{equation}\int_0^\infty e^{-x^2} dx=\frac{\sqrt{\pi}}{2}\label{eq:regular-equation}\end{equation}

An equation too long to fit within page:

\begin{equation}x = a + b + c + d + e + f + g + h + i + j + k + l + m + n + o + p + q + r + s + t + u + v + w + x + y + z + 1 + 2 + 3 + 4 + 5 + 6 + 7 + 8 + 9\label{eq:long-equation}\end{equation}

\hypertarget{special}{%
\subsection{Special}\label{special}}

{WARNING} \emph{The following features are only supported and intended for \texttt{.html} and \texttt{.pdf} exports.}
\emph{Journals are not likely to support them, and they may not display correctly when converted to other formats such as \texttt{.docx}.}

\href{https://manubot.org}{Link styled as a button}

Adding arbitrary HTML attributes to an element using Pandoc's attribute syntax:

\leavevmode\vadjust pre{\hypertarget{some_id_1}{}}%
Manubot Manubot Manubot Manubot Manubot.
Manubot Manubot Manubot Manubot.
Manubot Manubot Manubot.
Manubot Manubot.
Manubot.

Adding arbitrary HTML attributes to an element with the Manubot \texttt{attributes} plugin (more flexible than Pandoc's method in terms of which elements you can add attributes to):

Manubot Manubot Manubot Manubot Manubot.
Manubot Manubot Manubot Manubot.
Manubot Manubot Manubot.
Manubot Manubot.
Manubot.

Available background colors for text, images, code, banners, etc:

\texttt{white}
\texttt{lightgrey}
\texttt{grey}
\texttt{darkgrey}
\texttt{black}
\texttt{lightred}
\texttt{lightyellow}
\texttt{lightgreen}
\texttt{lightblue}
\texttt{lightpurple}
\texttt{red}
\texttt{orange}
\texttt{yellow}
\texttt{green}
\texttt{blue}
\texttt{purple}

Using the \href{https://fontawesome.com/}{Font Awesome} icon set:

{ \textbf{Light Grey Banner}
useful for \emph{general information} - \href{https://manubot.org/}{manubot.org}}

{ \textbf{Blue Banner}
useful for \emph{important information} - \href{https://manubot.org/}{manubot.org}}

{ \textbf{Light Red Banner}
useful for \emph{warnings} - \href{https://manubot.org/}{manubot.org}}

\hypertarget{kapitel-2-fragestellunghypothese}{%
\section{Kapitel 2: Fragestellung/Hypothese}\label{kapitel-2-fragestellunghypothese}}

In diesem Projekt soll untersucht werden, inwieweit CD44 als potenzieller Tumorstammzellmarker bei Patienten mit lokal fortgeschrittenen Kopf-Hals- Plattenepithelkarzinomen auch nach postoperativer Radiochemotherapie eine prognostische Rolle spielt. Darüber hinaus soll die CD44-Expression getrennt für die Patientenkollektive mit HPV16 DNA-positiven und -negativen Tumoren analysiert werden. In einer vorangegangenen Studie an der gleichen Patientenkohorte wurde gezeigt, dass der HPV16 DNA-Status ein Prognostikator für Patienten mit lokal fortgeschrittenen adjuvant behandelten Oropharynxkarzinomen ist (\href{about:blank\#_ENREF_52}{Lohaus et al., 2014}).

Dieses Projekt ist Teil einer multizentrischen Studie der Radioonkologie-Gruppe des Deutschen Konsortiums für Translationale Krebsforschung (DKTK-ROG) in dem die immunhistochemisch messbaren Biomarker für den Stammzellgehalt von Kopf-Hals-Plattenepithelkarzinomen (Patientenmaterial) angefärbt und evaluiert werden. Es erfolgte dann Korrelation mit den klinischen Ergebnissen der Radiochemotherapie (RCTx). Die Hypothese ist dass der gemessene Biomarker mit dem klinischen outcome korreliert. Mittelfristigs Ziel des Gesamtprojektes ist die Etablierung von Markern, die nach prospektiver Valedierung geeignet sind, als Grundlage individualisierte Therapieentscheidungen zu dienen.

Fragestellung/Hypothese

Das Hauptziel dieser Studie besteht darin, den Einfluss von CD44 als potenziellem Tumorstammzellmarker auf das Therapieansprechen und die Prognose bei Patienten mit lokal fortgeschrittenen Kopf-Hals-Plattenepithelkarzinomen nach postoperativer Radiochemotherapie zu untersuchen. Insbesondere wird untersucht, ob die Expression von CD44 mit dem Überleben, der Tumorkontrolle und der Entwicklung von Fernmetastasen korreliert. Zusätzlich sollen potenzielle Zusammenhänge zwischen der CD44-Expression und klinischen Faktoren wie Tumorgröße, Lymphknotenmetastasen und histologischer Differenzierung analysiert werden, um ein umfassendes Verständnis der prognostischen Bedeutung von CD44 zu erhalten.

Diese Forschung ist Teil einer umfangreichen multizentrischen Studie, die von der renommierten Radioonkologie-Gruppe des Deutschen Konsortiums für Translationale Krebsforschung (DKTK-ROG) durchgeführt wird. Das Ziel dieser Studie ist es, evidenzbasierte Erkenntnisse zur personalisierten Therapieentscheidung bei Patienten mit lokal fortgeschrittenen Kopf-Hals-Plattenepithelkarzinomen zu liefern. Durch die Identifizierung und Validierung von Biomarkern wie CD44 soll eine Grundlage für die Entwicklung individualisierter Behandlungsstrategie

Die Hypothese dieser Studie besagt, dass eine erhöhte Expression von CD44 mit einer schlechteren Prognose bei Patienten mit lokal fortgeschrittenen Kopf-Hals-Plattenepithelkarzinomen nach postoperativer Radiochemotherapie assoziiert ist. Es wird erwartet, dass Patienten mit einer hohen CD44-Expression ein erhöhtes Risiko für Tumorrezidive, regionale Lymphknotenmetastasen und Fernmetastasen aufweisen. Darüber hinaus wird vermutet, dass der HPV-Status einen Einfluss auf die prognostische Bedeutung von CD44 haben könnte, wobei HPV16-DNA-positive Tumoren möglicherweise eine höhere CD44-Expression und schlechtere klinische Ergebnisse aufweisen.

Um diese Hypothesen zu überprüfen, werden umfangreiche klinische Datenanalysen durchgeführt, bei denen die CD44-Expression in Gewebeproben mittels immunhistochemischer Methoden quantifiziert wird. Anschließend werden statistische Analysen durchgeführt, um potenzielle Zusammenhänge zwischen CD44-Expression und klinischen Endpunkten zu untersuchen. Multivariate Analysen werden angewendet, um den Einfluss anderer prognostischer Faktoren zu berücksichtigen und die unabhängige prognostische Bedeutung von CD44 zu bestimmen.

Die klinischen Daten werden aus einer großen Kohorte von Patienten mit lokal fortgeschrittenen Kopf-Hals-Plattenepithelkarzinomen nach postoperativer Radiochemotherapie gesammelt. Die Gewebeproben werden sorgfältig gesammelt und einer immunhistochemischen Färbung unterzogen, um die CD44-Expression zu quantifizieren. Die Ergebnisse dieser quantitativen Analyse werden dann in Beziehung zu den klinischen Endpunkten wie Überleben, Tumorkontrolle und Fernmetastasierung gesetzt.

Statistische Analysen werden durchgeführt, um potenzielle Zusammenhänge zwischen der CD44-Expression und den klinischen Endpunkten zu identifizieren. Hierbei werden auch andere prognostische Faktoren, wie zum Beispiel Tumorgröße, Lymphknotenmetastasen und histologische Differenzierung, berücksichtigt. Multivariate Analysen ermöglichen es, den unabhängigen Einfluss von CD44 auf die Prognose zu bestimmen, indem sie andere prognostische Faktoren kontrollieren.

Diese umfangreichen Datenanalysen werden von erfahrenen Biostatistikern unter Berücksichtigung der neuesten statistischen Methoden durchgeführt. Die Ergebnisse werden anschließend sorgfältig interpretiert, um die prognostische Bedeutung von CD44 bei Patienten mit lokal fortgeschrittenen Kopf-Hals-Plattenepithelkarzinomen nach postoperativer Radiochemotherapie zu bestimmen.

Innerhalb eines größeren, von Dresden geleiteten multizentrischen Projekts

\hypertarget{kapitel-3-material-und-methoden}{%
\section{Kapitel 3: Material und Methoden}\label{kapitel-3-material-und-methoden}}

\hypertarget{studiendesign}{%
\subsection{Studiendesign}\label{studiendesign}}

Diskussion des Studiendesigns

Für die Erfolgreiche Durchführung von Biomarker-Studien werden ausreichend große Patientenkohorten benötigt, die mit aktuellen Therapiestrategien beahndelt werden. Durch eine multizentrische nationale oder internationale Ausrichtung von Studien können eine höhere Rekrutierungszahl erreicht und die Zeiträume vo Studien verkürzt werden, sodass im Falle der retrospektiv durchgeführten Studien die Therapie der aktuellen Standardtherapie entspricht. Die multizentrisch generierten Studienergebnisse sind dabei besonders robust, weil Einzelzentrumseffekte und anderer Sektionsbias damit reduziert werden.

Das Deutsche Konsortium für Translationale Krebsforschung (DKTK) ist ein nationales Forschungsnetzwerk, das im Oktober 2012 gegründet wurde (\href{about:blank\#_ENREF_41}{Joos et al., 2019}). In der Radioonkologie-Gruppe des DKTK (DKTK-ROG) sind alle 8 Partnerstandorte des DKTK organisiert. Im Rahmen der ersten gemeinsamen Studie der DKTK-ROG, sollen prognostische und prädiktive Biomarker für die lokoregionäre Kontrolle nach primärer bzw. nach postoperativer Radiochemotherapie in Patienten mit lokoregionären fortgeschrittenen Kopf-Hals-Plattenepithelkarzinom zunächst im retrospektiven Teil der Studie identifiziert, in dem prospektiven Studienteil validiert und in einer Interventionsstudie angewendet werden. Mittelfristiges Ziel des Gesamtprojektes ist die Etablierung von Markern, die nach prospektiver Validierung geeignet sind, als Grundlage individualisierter Therapieentscheidungen zu dienen. Ein aus den Studienleitern der jeweiligen Standorte gebildeter Lenkungsausschuss und die gemeinsame Co-Ownerschaft stellen den hohen Standard der Studie sicher. Die Studie wird durch die DKTK-ROG-Studienzentrale in Dresden betreut.

Die in dieser Arbeit vorgestellten Ergebnisse sind Teil der retrospektiven Untersuchung der multizentrischen Studie, in der immunhistochemisch messbare Biomarker für den Stammzellgehalt von Kopf-Hals-Plattenepithelkarzinomen angefärbt und evaluiert wurden. Anschließend erfolgte die Korrelation mit den klinischen Ergebnissen der postoperativen Radiochemotherapie.

\hypertarget{patientenkollektiv-und-tumormaterial}{%
\subsection{Patientenkollektiv und Tumormaterial}\label{patientenkollektiv-und-tumormaterial}}

Das Patientenkollektiv bestand aus 221 Patienten mit lokal fortgeschritten Plattenepithelkarzinom des Kopf-Hals-Bereichs, die im Zeitraum zwischen 2004 und 2012 in den 8 Partnerzentren des Deutschen Konsortiums für Translationale Krebsforschung (DKTK) behandelt wurden. Die Ethikkommissionen aller DKTK-Partnerstandorte haben die ethische Genehmigung für multizentrische retrospektive Analysen klinischer und biologischer Daten eingeholt (AZ EK299092012).

Einschlusskriterien waren ein histologisch gesichertes Plattenepithelkarzinom der Mundhöhle, des Oro- oder Hypopharynx. Alle Patienten hatten nach der primären onkologischen Tumorresektion aufgrund des hohen Rezidivrisikos (Tumorstadium pT4 und/oder mikroskopischer Tumorrest nach Operation und/oder kapselüberschreitendes Wachstum der Lymphknotenmetastasen oder \textgreater3 Lymphknotenmetastasen) an einem der 8 Partnerstandorte des DKTK eine postoperative Cisplatin-basierte Radiochemotherapie (PORT-C) gemäß Standardprotokolle in kurativer Intension erhalten. Darüber hinaus war für den Studieneinschluss, neben klinischen Parametern unter anderem auch das Vorhandensein von FFPE-Tumormaterial (vor jeder tumorspezifischen Behandlung), Bestrahlungspläne sowie die Daten des klinischem follow-up mit Bildgebung / bildgebender Diagnostik (CT-, MRT- oder PET-CT) zur späteren Lage-Beurteilung eines möglichen Tumorrezidivs verpflichtend und wurden zentral am DKTK-Standort Dresden gesammelt (\href{about:blank\#_ENREF_52}{Lohaus et al., 2014}). Die analysierten Gewebeproben entstammten überwiegend aus Primärresektaten, wobei im Falle von fehlendem Primärtumorresektaten in wenigen Fällen auf Tumorbiopsien, die im Rahmen einer diagnostischen Planendoskopie entnommen wurden, oder Lymphknotenmetastasen zurückgegriffen wurde. Die Mindestnachbeobachtungszeit musste 24 Monate betragen (Letzte Rekrutierung 2010). Der Raucherstatus und Alkoholkonsum wurden nicht für alle Patienten konsistent aufgezeichnet und konnten daher nicht analysiert werden. Alle Einschlusskriterien sind in Tabelle 4‑1 nochmals übersichtlich dargestellt. Die Patientenmerkmale sind in der Tabelle 5‑1 dargestellt.

Tabelle 4‑1: Einschlusskriterien für Patienten

\begin{tablenos:no-prefix-table-caption}

\begin{longtable}[]{@{}
  >{\raggedright\arraybackslash}p{(\columnwidth - 2\tabcolsep) * \real{0.5000}}
  >{\raggedright\arraybackslash}p{(\columnwidth - 2\tabcolsep) * \real{0.5000}}@{}}
\toprule()
\begin{minipage}[b]{\linewidth}\raggedright
Einschlusskriterien für Patienten
\end{minipage} & \begin{minipage}[b]{\linewidth}\raggedright
\end{minipage} \\
\midrule()
\endhead
Histologie & Histologisch gesichertes Plattenepithelkarzinom \\
Tumorlokalisation & Mundhöhlen-, Oro- oder Hypopharynxkarzinom \\
Behandlung & Zwischen 2004 und 2012 \\
Lokal fortgeschrittene Erkrankung & Kurativ intendierte Cisplatin-basierte Radiochemotherapie der Tumorregion und der Halslymphknoten \\
Follow-up & mindestens 24 Monate ohne fortschreitende Erkrankung \\
Biomaterial & FFPE-Material \\
Therapiepläne und Bildgebung & Bestrahlungspläne, CT, MRT oder PET-CT \\
\bottomrule()
\end{longtable}

\end{tablenos:no-prefix-table-caption}

Nachträglich wurden 26 Patienten aufgrund von unzureichendem Biomaterial oder Nichterfüllen der Einschlusskriterien ausgeschlossen. Folglich waren für das vorliegender Projekt 195 in Formalin fixierte und in Paraffin eingebettete (FFPE) Operationspräparate verfügbar die nach standardisierten Verfahren am DKTK-Partnerstandort Dresden verarbeitet wurden und für die Biomarkeranalyse zur Verfügung standen.

\hypertarget{immunhistologische-analysen}{%
\subsection{Immunhistologische Analysen}\label{immunhistologische-analysen}}

\hypertarget{tissue-microarray}{%
\subsubsection{Tissue-Microarray}\label{tissue-microarray}}

Die Tissue-Microarray Technik wurde erstmals 1998 beschrieben (\href{about:blank\#_ENREF_46}{Kononen et al., 1998}) und ermöglicht eine effiziente und auf den Tumor fokussierte simultane Auswertung der einzelnen Gewebeproben. Die Methode hat seither die Art und Weise wie gewebebasierte Forschung, insbesondere im onkologischen Bereich stattfindet maßgeblich geprägt. Die TMA-Technik ermöglicht es, Gewebsproben verschiedener Patienten bzw. verschiedene Gewebeproben eines Patienten hinsichtlich ihrer immunhistochemischen Proteinexpression simultan und unter gleichen methodischen Bedingungen standardisiert zu untersuchen. Ein TMA-Block kann bis zu 150 - 200 zylindrisches Probematerial von verschiedenen Patienten mit einem Durchmesser von jeweils 0,6 -- 1,0 mm enthalten. Die Gewebezylinder werden mit einer Hohlnadel aus einem vorher ausgewählten repräsentativen Bereich eines Spender-Paraffinblocks entnommen und danach in vorgefertigte Löcher in einem ursprünglich leeren Empfänger-Paraffinblock eingebracht.

\emph{Herstellung der Referenzschnitte (HE-Färbung)}

Um das Vorhandensein eines Plattenepithelkarzinoms histologisch zu bestätigen und um repräsentative Tumorregionen im FFPE-Material für die zu untersuchenden Stanzungen zu definieren, wurden für alle FFPE-Tumorproben des Kollektivs zunächst Objektträger mit Gewebeschnitten in Hämatoxylin- und Eosinfärbung (HE-Färbung) angefertigt. Dabei wurden je 3 μm Schnitte der 195 zu untersuchenden FFPE-Blöcke mit einem Mikrotom angefertigt und auf Glasobjektträger aufgezogen und anschließend 24 Stunden bei Raumtemperatur getrocknet. Danach wurden die Schnitte in einer absteigenden Alkoholreihe deparaffiniert (Xylol, Propanol, Ethanol 96\%, Ethanol 70\%, Aqua dest.) und in TRIS-Puffer gewaschen. Nach der Inkubation mit Hämatoxylin und anschließendem Waschen mit Leitungswasser wurden die Schnitte mit Eosin inkubiert. Durch eine aufsteigende Alkoholreihe erfolgte die Entwässerung. Zum Schluss wurden die Schnitte mit Deckgläsern versehen. Diese Referenzschnitte wurden histologisch beurteilt und die repräsentativen Tumorregionen markiert (HNSCC, Lymphknoten Gewebe und normales Plattenepithel der Kopf-Hals Bereiches). Anschließend erfolgte die Übertragung der Markierung auf die (Spender-) Paraffinblöcke.

\emph{Herstellung der Tissue Microarray Blöcke für die Biomarkeranalyse}

Nachdem die mikroskopisch repräsentativen Bereiche der in Paraffin eingebetteten Operationsresektate entsprechend der HE-Referenzpräparate auf dem Spender-Block übertragen wurden, erfolgte die Anfertigung von Gewebe-Mikroarrays (tissue micro arrays, TMA). Dazu wurden mit einem semiautomatischen Gewebe-Mikro-Arrayer (MiniCore®, Alphelys) 1 mm dicke Gewebezylinder über eine Hohlnadel aus vorher ausgewählten repräsentativen Tumorarealen im Spenderblock ausgestanzt. Diese Gewebezylinder wurden in leere Stanzlöcher eines Empfängerblocks überführt. Idealerweise wurden je Fall drei Stanzzylinder entnommen, um die Heterogenität innerhalb des Tumors zu berücksichtigen. Somit war es möglich pro Trägerblock bis zu 112 Stanzen einzubringen. Durch Erhitzung auf 60°C über 5 Minuten konnte ein Verbund des Träger- und Gewebeparaffins erreicht werden. Die Anordnung der Stanzzylinder im Empfängerblock erfolgte über einen zuvor definierten Lageplan, der für die spätere Auswertung eine eindeutige Zuordnung zum pseudonymisierten Patienten gewährleistet. In Abbildung sind der Aufbau und das Prinzip der TMA-Technik sowie der Dokumentation beispielhaft dargestellt.

\begin{fignos:no-prefix-figure-caption}

\begin{figure}
\centering
\includegraphics{Abbildungen/image5.jpeg}
\caption{Abbildungen/image5.jpeg}
\end{figure}

\end{fignos:no-prefix-figure-caption}

Abbildung 4‑1: Aufbau der Tissue Microarray-Technik sowie Prinzip der Ortsdokumentation der einzelnen Tumor-Spots

\begin{fignos:no-prefix-figure-caption}

\begin{figure}
\centering
\includegraphics{Abbildungen/image6.jpeg}
\caption{Abbildungen/image6.jpeg}
\end{figure}

\end{fignos:no-prefix-figure-caption}

\hypertarget{immunhistologische-reaktionen}{%
\subsubsection{Immunhistologische Reaktionen}\label{immunhistologische-reaktionen}}

Die immunhistologische Färbereaktion wurde nach der indirekten Avidin-Biotin-Komplex- (ABC)-Methode durchgeführt.

Für die immunhistochemische Färbung wurden von jedem TMA-Block jeweils 3 μm dicke Schnitte angefertigt. Diese wurden in Xylol entparaffiniert und anschließend in einer absteigenden Alkoholreihe rehydriert. Für die anschließende Demaskierung der Primärantikörper-Bindungsstellen wurden die Objektträger in Zitratpuffer (pH 6; Dako) für 35 min bei 650W in einer Mikrowelle erhitzt und nach dem Abkühlen mit Waschpuffer gespült. Anschließend erfolgte die Blockierung der endogenen Peroxidaseaktivität (DAKO RealTM Peroxidase Blocking Solution; 10 min). Die TMA-Schnitte wurden mit dem monoklonalen Primärantikörper \emph{Maus anti-human CD44-Antikörper} (Clone DF 1485; Verdünnung 1:100; Dako) für eine Stunde bei Raumtemperatur in einer feuchten Kammer inkubiert. Die Negativkontrolle wurde entsprechend mit IgG1-Kontrollreagenz (Mouse IgG1; Dako) inkubiert. Nach Zugabe des Visualisierungs-reagenz (Dako REAL EnVision) und anschließender Spülung der Objektträger mit Waschpuffer ergab sich die Farbreaktion durch Zugabe von DAB-Chromogenlösung entsprechend der Antigenverteilung im Gewebe. Die Gegenfärbung der Schnitte erfolgte mit Hämalaun (nach Mayer). Abschließend wurden die gefärbten Schnitte in aufsteigenden Alkohollösungen und Xylol dehydriert und mit Eukitt eingedeckelt.

\hypertarget{immunhistologische-auswertung}{%
\subsubsection{Immunhistologische Auswertung}\label{immunhistologische-auswertung}}

Die lichtmikroskopische Auswertung der immunhistochemisch gefärbten TMA-Schnitte erfolgte durch zwei unabhängige Untersucher (A.L. und C.v.N.; \textless5\% Abweichung zwischen den Ergebnissen beider Untersucher). Die Auswertung erfolgte semiquantitativ bei 100-, 200- und 400-facher Vergrößerung am Axioskop 50 (Zeiss, Germany) unter Berücksichtigung der Färbeintensität (negativ, +, ++). Jede Gewebestanze wurde zunächst separat bewertet. Als nicht auswertbar galten neben nicht vorhandenen Stanzen solche, die nicht oder nur zu einem kleinen Teil (\textless10\%) aus Tumorgewebe bestanden, außerdem jene, welche sich inhomogen anfärbten. Bei mehreren auswertbaren Stanzen pro Patientenprobe wurde nur die stärkste Färbeintensität berücksichtigt. Karzinome mit mäßiger oder starker CD44-Expression (+, ++) wurden als CD44-positiv angesehen und Karzinome mit nicht nachweisbarer CD44-Expression (0) wurden als CD44-negativ für die weitere statistische Auswertung berücksichtigt. Ein Tumor galt insgesamt als nicht bewertbar, wenn weniger als zwei auswertbare Stanzen vorhanden waren. Nicht bewertbare Tumoren gingen nicht in die Statistik mit ein. Insgesamt waren jedoch TMA-Kerne von 195 Patienten hinsichtlich der CD44-Proteinexpression auswertbar.

\begin{fignos:no-prefix-figure-caption}

\begin{figure}
\centering
\includegraphics{Abbildungen/image6.jpeg}
\caption{Abbildungen/image6.jpeg}
\end{figure}

\end{fignos:no-prefix-figure-caption}

\textbf{Abbildung 4‑2: Immunhistochemische Färbung von CD44 mit unterschiedlichen Färbeintensitäten. Alle Färbeintensitäten (+, ++, +++) wurden als positive Färbung angesehen}

Mit Hilfe des Tissue-Microarray(TMA)-Verfahrens konnte das Karzinomgewebe (n = 169) auf 11 TMA- Blöcke überführt werden. Davon wurden 10 für das Tonsillen-Karzinomgewebe verbraucht. Das Gewebe der Lymphknotenmetastasen wurde auf einen separaten TMA-Block überführt. Insgesamt lagen 1.401 Gewebestanzen vor. Von insgesamt 169 Fällen waren 166 (98,2\%) auswertbar. Bei den verbliebenen 3 Fällen (0,8\%) ließ sich kein Tumorgewebe im Schnittpräparat diagnostizieren. Das Gewebe der Lymphknotenmetastasen war in 96\% (n = 26) der Fälle auswertbar. Die geringe Anzahl der Ausfälle kam dadurch zustande, dass jedes Karzinomgewebe pro Fall dreimal gestanzt wurde.

\hypertarget{statistische-methoden-und-klinische-endpunkte}{%
\subsection{Statistische Methoden und klinische Endpunkte}\label{statistische-methoden-und-klinische-endpunkte}}

Zunächst wurden alle Daten mit dem Tabellenkalkulationsprogramm Excel erfasst. Nach Verschlüsselung der Patientenidentitäten wurden diese anonymisiert in das SPSSS-Software-Programm (Statistical Package for the Social Science Version 28.0 für Mac OS) übernommen, mit dem die statistische Auswertung der gesammelten Daten erfolgte.

Der Beobachtungszeitraum wurde für alle Ereigniszeitanalysen wurde vom ersten Tag der Strahlentherapie bis zum Datum des Ereignisses oder der Zensur berechnet. Die Mindestnachbeobachtungszeit der einzelnen Patienten musste (gemäß der Einschlusskriterien) 24 Monate betragen, um in der statistischen Datenauswertung berücksichtigt zu werden. Der primäre Endpunkt war die lokoregionäre Kontrolle (LRC) und sekundäre Endpunkte waren das metastasenfrei Überleben (MFS) und das Gesamtüberleben (GS). Die verscheidenn klinischen Endpunkte wurden wie folgt definiert:

\begin{tablenos:no-prefix-table-caption}

\begin{longtable}[]{@{}
  >{\raggedright\arraybackslash}p{(\columnwidth - 2\tabcolsep) * \real{0.5000}}
  >{\raggedright\arraybackslash}p{(\columnwidth - 2\tabcolsep) * \real{0.5000}}@{}}
\toprule()
\begin{minipage}[b]{\linewidth}\raggedright
Klinischer Endpunkt
\end{minipage} & \begin{minipage}[b]{\linewidth}\raggedright
Definition
\end{minipage} \\
\midrule()
\endhead
Lokoregionäre Tumorkontrolle & Zeitraum zwischen der Radiochemotherapie und dem Auftreten eines Lokalrezidivs (Ereignis) bzw. das zuletzt erfasste Datum der Nachbeobachtung, an dem das Ereignis nicht eingetreten war \\
Fernmetastasenfreies Überleben & Zeitraum zwischen Abschluss der Radiochemotheraphie und dem Auftreten von Fernmetastasen bzw. das zuletzt erfasste Datum der Nachbeobachtung, an dem das Ereignis nicht eingetreten war \\
Gesamtüberleben & Zeitraum zwischen Abschluss der Radiochemotherapie und dem tumorbedingten Todeszeitpunkt / Tod jedweder Art~bzw. der zuletzt erfasste Zeitpunkt, an dem die Patienten am Leben waren \\
\bottomrule()
\end{longtable}

\end{tablenos:no-prefix-table-caption}

Der Endpunkt für das Gesamtüberleben (OS) war entweder das tumorbedingte Sterbedatum oder der zuletzt erfasste Zeitpunkt, an dem die Patienten am Leben waren. Die Endpunkte für das metastasenfreie Überleben (MFS) und die lokoregionäre Kontrolle (LRC) waren entweder das Diagnosedatum der Metastase bzw. des Rezidivs oder das zuletzt erfasste Datum der Nachbeobachtung, zu dem keines der beiden Ereignisse eingetreten war. Die Diagnosesicherung des Rezidivs erfolgte hierbei pathologisch nach bioptischer Probenentnahme.

Die Dauer der Ereignisfreien Zeit / Überelebnseiz (Follow Up) wurde für alle o.g. Endpunkte mithilfe der Kaplan-Meier-Methode geschätzt und in entsprechende Überlebenskurven grafisch dargestellt. Für alle Endpunkte wurden die entsprechenden Überlebsraten und (wenn möglich) mediane Überlebszeit angegeben. Die berechneten Zeitwerte der Überlebenswahrscheinlichkeit wurden, wenn möglich jeweils als Median gemessen.

Bei der Analyse von Überlebenszeitdaten wird das Kaplan-Meier-Verfahren benutzt. Es können Überlebensraten und die mediane Überlebenszeit angegeben werden. Mit Hilfe des Log-rank-Tests kann man die Überlebenszeiten von zwei Gruppen miteinander vergleichen. Für multivariable Modelle verwendet man die Cox-Regression. Das Hazard Ratio als deskriptives Maß für den Unterschied von Überlebenszeiten wird erläutert.

Nachbeobachtungszeit wurde vom Zeitpunkt der primären Tumorresektion berechnet und die Überlebsnwahrscheinlichkeit für alle Endpunkte mit hilfe der Kaplan maier methode geschätzt.

Die Überlebenszeitanalysen wurden mit dem univariaten Verfahren nach Kaplan- Meier durchgeführt und die geschätzten Überlebenszeiten in Kaplan-Meier Ereigniskurven dargestellt

✍ Kaplan meier kurven daruas überlebsraten und mediane überlebesnzeit wenn weniger alks bei der hälfte ein ereignis eigetreten wa

Falls der Kaplan-Meier-Schätzer in der gesamten Beobachtungszeit über 50 \% liegt, ist die mediane Überlebenszeit nicht zu bestimmen. In diesem Fall ist bis zur maximalen Beobachtungszeit für weniger als die Hälfte der Patienten ein Ereignis eingetreten.

Die mediane Überlebenszeit bzw. mediane ereignisfreie Zeit kann man nur bestimmen, falls der Kaplan-Meier Schätzer unter 0,5 fällt. Falls im Beobachtungszeitraum bei weniger als 50\% der Klienten ein Ereignis eingetreten ist, kann die mediane Überlebenszeit nicht angegeben werden.

Als Alternative zur medianen Überlebenszeit wird irrtümlich manchmal statt der medianen Überlebenszeit die mittlere oder durchschnittliche Überlebenszeit angegeben. Diese Maßzahl ist aber bei Vorliegen von Zensierungen unbrauchbar und nicht mehr zu interpretieren. Vorsicht ist vor allem deshalb geboten, da diese Maßzahl standardmäßig von vielen Programmen berechnet wird.

Die Berechnung der statistischen Signifikanz der Unterschiede zwischen den Überlebenskurven zweier Subgruppen erfolgte mit dem \textbf{Log Rank} Test, bei dem das Signifikanzniveau auf 5\% (p\textless0,05) gesetzt wurde. Ein signifikanter Unterschied wurde somit bei einem p-Wert von \textless0,05 (Kaplan-Meier-Methode) und \textless0,1 (Cox-Regression) angenommen. Bei einem p\textless0,05 (Log Rank) in der Kaplan-Meier-Analyse wurde ggf. eine multivariate Analyse mittels Cox-Regression durchgeführt, um die Signifikanz zu validieren.

Der Einfluss potenzieller prognostischer Variablen auf die Endpunkte wurde mit dem univunivariaten \textbf{Cox-Regressionsmodell} bewertet. Parameter, die in der univariaten Analyse als signifikant befunden wurden, wurden in ein multivariates Cox-Modell aufgenommen. Bei dieser Methode wurde ein signifikanter Unterschied bei einem p-Wert von \textless0,1 angenommen.

Das Risiko am Tumor zu versterben wurde schließlich in der so genannten Hazard Ratio (HR) ausgedrückt. Bei einer HR=1,0 für eine Patientengruppe gab es keinen Unterschied zur Referenzgruppe; bei einer HR\textless1,0 hatte die Patientengruppe im Vergleich zur Referenzgruppe ein besseres Überleben und bei einer HR\textgreater1,0 hatte die Patientengruppe im Vergleich zur Referenzgruppe ein schlechteres Überleben.

Ein Signifikanter P- Wert bedeutet das die Kovariate einfluss auf das Überleben, die Ereigniszeit hat.

Außerdem gingen folgende Parameter in die univariate Analyse ein:

\begin{itemize}
\tightlist
\item
  Klinische Prognosefaktoren

  \begin{itemize}
  \tightlist
  \item
    T-Stadium des Tumors
  \item
    Differenzierungsgrad des Tumors
  \item
    Status des Resektionsrandes:
  \item
    Zeitraum zwischen OP und adjuvanter Strahlentherapie (Mittelwert = 63 Tage)
  \item
    Tumorlokalisation: Mundhöhle, Oropharynx, Hypopharynx, Larynx
  \item
    UICC -- Stadium des Tumors
  \end{itemize}
\item
  Pathologische/biologische Prognosefaktoren

  \begin{itemize}
  \tightlist
  \item
    HPV16-DNA-Status
  \item
    CD44-Expressions-Status
  \end{itemize}
\end{itemize}

den immunhistochemischen Färbeergebnissen der CD44-Proteinexpression, dem HPV16-DNA-Status, der histopathologischen Differenzierungsgraden, der Tumorlokalisation, der Tumorausdehnung (pT-Kategorie), der Stadiengruppierung der Karzinome bezüglich der TNM-Kategorien (UICC-Stadien), des Alters (50-54, 55-59, 60-64,\ldots), der Nikotin- und Alkoholanamnese

Statistische Analysen wurden für alle Patienten und für die \textbf{Untergruppe} der Patienten mit Mundhöhlenkrebs sowie Oropharyngealkarzinomen durchgeführt. Patienten, bei denen ein hypopharyngeales Karzinom diagnostiziert wurde, wurden aufgrund der geringen Anzahl von Fällen von dieser Untergruppenanalyse ausgeschlossen. Für eine weitere Stratifizierung wurden statistische Analysen für die Subgruppen der Patienten mit HPV16 DNA-negativen Tumoren durchgeführt. In der Untergruppe der Patienten mit HPV16 DNA-positiven Tumoren traten nur zwei Rezidiven auf. Daher war es nicht möglich, signifikante Unterschiede in der LRC für diese Teilgruppe zu erkennen.

Patienten mit HPV-assoziierten Kopf-Hals-Plattenepithelkarzinome besitzen ein insgesamt besserer klinisches Outcome als Patienten mit HPV-negativen Tumoren (\href{about:blank\#_ENREF_21}{Chen et al., 2010}; \href{about:blank\#_ENREF_32}{Fakhry et al., 2008}; \href{about:blank\#_ENREF_35}{Gillison et al., 2000}; \href{about:blank\#_ENREF_49}{Lindel et al., 2001}; \href{about:blank\#_ENREF_52}{Lohaus et al., 2014}; \href{about:blank\#_ENREF_65}{Shi et al., 2009}; \href{about:blank\#_ENREF_75}{Weinberger et al., 2006}). Deshalb wurden univariate Ereignisdatenanalysen hinsichtlich des Einflusses der CD44-Proteinexpression, sowohl für die loko-regionäre Tumorkontrolle als auch für das Fernmetastasen-freie Überleben und Gesamtüberleben stratifiziert nach den viralen Marker HPV16 DNA analysiert, um das diagnostische und prognostische Potential dieser Marker zu prüfen.

Die Sensitivität und Spezifität der CD44-Expression zur Vorhersage eines lokoregionären Rezidivs wurden durch Kreuztabellen bestimmt.

die für die Tumorlokalisierung geschichtet wurden

Zur Vorhersage des 2-Jahres-LRC wurde eine multivariate logistische Regression durchgeführt

Um die Überlebenskurven von Patienten zu vergleichen, die nach HPV16-DNA-Status, CSC-Markerexpression und Hypoxiestatus geschichtet waren, wurden Log-Rank-Tests durchgeführt

Konfindnenzintervall = Vertrauenswahrscheinlichkeit

Statistische Analysen wurden für alle Patienten und für die Untergruppen von Patienten mit Mundhöhlenkrebs sowie Oropharyngealkarzinomen durchgeführt. Patienten, bei denen~\href{https://www.sciencedirect.com/topics/medicine-and-dentistry/hypopharyngeal-cancer}{Hypophoaryngealkrebs}~diagnostiziert wurde, wurden aufgrund der geringen Anzahl von Fällen von dieser Untergruppenanalyse ausgeschlossen.

Die Einteilung der Ordinaten der Kaplan-Meier-Plots erfolgte von 0,00 in Zweierschritten bis 1,00, zur besseren Übersicht in einigen Abbildungen von 0,5 in Einerschritten bis 1,00. Berechnungsziele waren die allgemeine Überlebensrate, die krankheitsfreie Überlebensrate (Patientinnen ohne Metastasen oder lokoregionäres Rezidiv) und die lokale Kontrolle. Die Kaplan-Meier-Kurven wurden mit Hilfe des Log-Rank-Tests auf Signifikanz geprüft.

Weitere Analysen erfolgten nach dem univariaten Cox-Regression-Modell. Positive Regressionskoeffizienten verringern die Ereigniswahrscheinlichkeit, negative vergrößern diese. Der Exp(B) ist der Wert, um den sich das Risiko für ein Ereignis verändert, wenn die untersuchte Variable um eine Einheit steigt. Ein Wert von eins lässt das Risiko unverändert. Ist der Wert \textgreater1 steigt das Risiko für ein Ereignis, ist der Wert \textless1 sinkt das Risiko.

16

Multivariate Analysen wurden mit dem Cox-Regressions-Modell berechnet. Dabei wurden die signifikanten Variablen der univariaten Analyse in das Modell eingeschlossen. Die Reduktion der Variablen erfolgte über eine schrittweise Rückwärts-Prozedur nach der Wald-Methode.

Häufigkeiten wurden mit Kreuztabellen und dem Chi-Quadrat-Test bewertet. Waren die Voraussetzungen für den Chi-Quadrat-Test nicht gegeben, wurde der exakte Test nach Fisher verwendet. Der Vergleich von Mittelwerten kontinuierlicher Größen erfolgte mit dem zweiseitigen t-Test.

Die deskriptiven Daten wurden mittels absoluter und relativer Häufigkeit sowie Mittelwert, Median und Standardabweichung dargestellt. Mit Hilfe von Kreuztabellen wurde die relative Häufigkeit der möglichen Einflussfaktoren auf die Entstehung des atopischen Ekzems (Alter, Geschlecht, Bildungsstand der Eltern, Wohnort) bivariat ermittelt und mittels Chi-Quadrat-Test auf Unabhängigkeit überprüft.

Mit der multiplen logistischen Regressionsanalyse wurde anschließend der Zusammenhang zwischen allen potenziellen Einflussfaktoren und den untersuchten Symptomen ermittelt.

Zeitdauer zwischen Therapiebeginn und Rezidiv eines Tumors

Haben Tumorstammzellen ein Einfluss auf das tumorfreie Überleben

Für die deskriptive Statistik wurden Häufigkeiten berechnet und Kreuztabellen erstellt. Mithilfe der Kaplan-Meier-Methode wurden die Überlebens- kurven für das Gesamtüberleben und das rezidivfreie Überleben ermittelt. Das Gesamt- überleben betrachtet die Zeitspanne zwischen Diagnosestellung und Tod. Patienten, die am Ende der Beobachtungszeit noch am Leben waren, wurden zensiert. Das rezidiv- freie Überleben betrachtet die Zeitspanne zwischen Diagnosestellung und Datum des Rezidivs, das bedeutet Lokalrezidiv oder Fernmetastasen. Patienten, die kein Rezidiv erlitten oder verstarben, wurden zensiert.

Es wurden verschiedene klinische und biologische Faktoren bezüglich ihrer Auswirkung auf die jeweiligen Endpunkte überprüft und die jeweiligen Ereigniszietzen mit der Kaplan-Maier-Methode geschätzt.

Überlebensraten (Überlebenswahrscheinlichkeiten) wurden für folgende 17 Faktoren vergleichend berechnet: gesamt, Alter, Geschlecht, Lokalisationen, Stadium, Therapiekategorien, T-Stadium, N-Stadium, M-Stadium, Operation, Bestrahlung, Chemotherapie und Rezidiv. Prognostische Faktoren für das Gesamtüberleben und das rezidivfreie Überleben wurden in univariaten Analysen mithilfe des Log-Rank-Tests ermittelt. Als statistisch signifikant galt ein p-Wert von kleiner oder gleich 0,05. Hier wurden Alter, Geschlecht, Lokalisationen, Stadium, T-Stadium, N-Stadium, M-Stadium, Therapiekategorien, Operation, Bestrahlung, Chemotherapie untersucht. Für das Gesamtüberleben wurde außerdem der Parameter Rezidiv betrachtet. Statistisch signi- fikante prognostische Faktoren wurden einer Regressionsanalyse nach Cox zugeführt. In dieser multivariaten Analyse wurde ein p-Wert von kleiner oder gleich 0,05 als statistisch signifikant gewertet. Hierbei wurden folgende Parameter in die Analyse ein- geschlossen: Alter, Geschlecht, Lokalisationen, Stadium, T-Stadium, N-Stadium,

M-Stadium, Therapiekategorien, Operation, Bestrahlung, Chemotherapie.

\hypertarget{kapitel-4-ergebnisse}{%
\section{Kapitel 4: Ergebnisse}\label{kapitel-4-ergebnisse}}

\hypertarget{deskriptive-analyse-der-studienpopulation}{%
\subsection{Deskriptive Analyse der Studienpopulation}\label{deskriptive-analyse-der-studienpopulation}}

Im Folgenden wird zunächst auf die deskriptive Statistik eingegangen, hiernach erfolgt die Untersuchung der Überlebenszeitdaten bzw. Ereigniszeitdaten.

\hypertarget{patientenmerkmale-und-tumorcharakteristik}{%
\subsubsection{Patientenmerkmale und Tumorcharakteristik}\label{patientenmerkmale-und-tumorcharakteristik}}

Im Rahmen dieser multizentrischen retrospektiven Studie wurden 221 Patienten, die wegen eines lokal fortgeschrittenen Kopf-Hals-Plattenepithelkarzinoms eine postoperative Radiochemotherapie erhalten haben, untersucht. Die allgemeinen Patientenmerkmale sind in Tabelle 5‑1 zusammengefasst. Das mediane Alter zu Beginn der Strahlentherapie lag bei 56,5 Jahren. Die Altersspanne reichte von 27 bis 76 Jahre. Das Patientenkollektiv bestand zu 81,4\% (n=180) aus Männern und zu 18,6\% (n=41) aus Frauen. Die Informationen bezüglich des Zigaretten- und Alkoholkonsums war nur inkonsequent von allen Patienten eruierbar. Ein Großteil der Patienten (48,4\%) gab regelmäßigen Tabakgebrauch während der Behandlung oder in der Vergangenheit an. Aussagen zum regelmäßigen Alkoholkonsum vor und während des Beobachtungszeitraum wurden nicht erhoben.

\textbf{Tabelle 5‑1: Klinisch-pathologische Daten des Patientenkollektivs}

\begin{tablenos:no-prefix-table-caption}

\begin{longtable}[]{@{}llll@{}}
\toprule()
& & n & \% \\
\midrule()
\endhead
Anzahl der Patienten & & 221 & \\
Alter & 20-30 & 1 & 0.5 \\
& 31-40 & 7 & 3.2 \\
& 41-50 & 43 & 19.5 \\
& 51-60 & 79 & 35.7 \\
& 61-70 & 69 & 31.2 \\
& 71-80 & 8 & 3.6 \\
& Fehlend & 15 & 6.8 \\
Geschlecht & Männlich & 180 & 81.4 \\
& Weiblich & 41 & 18.6 \\
Nikotin & Raucher & 107 & 48,4 \\
(im Beobachtungszeitraum) & Nichtraucher & 99 & 44,8 \\
& Fehlend & 15 & 6,8 \\
\bottomrule()
\end{longtable}

\end{tablenos:no-prefix-table-caption}

\url{Abbildungen/image7.emf}

Bei allen im Kollektiv eingeschlossenen Patienten konnte ein lokal fortgeschrittenes Plattenepithelkarzinom im Kopf-Hals-Bereich diagnostiziert werden, was durch die jeweiligen pathologischen Institute der DKTK-Partnerstandorte histologisch bestätigt und klassifiziert wurde. Bei 60 Patienten war der Tumor in der Mundhöhle (27,1\%) lokalisiert, bei 126 Patienten im Oropharynx (57,0\%) und bei 35 Patienten im Hypopharynx (15,8\%). Die Anzahl der Patienten pro DKTK-Behandlungszentrum und Tumorlokalisierung sind in Tabelle 5‑2 zusammengefasst.

\textbf{Tabelle 5‑2: Anzahl der Patienten pro Behandlungszentrum und Tumorlokalisierung}

\begin{tablenos:no-prefix-table-caption}

\begin{longtable}[]{@{}lllll@{}}
\toprule()
Behandlungszentrum & n & Mundhöhle & Oropharynx & Hypopharynx \\
\midrule()
\endhead
Berlin & 24 & 9 & 11 & 4 \\
Dresden & 42 & 21 & 14 & 7 \\
Essen & 33 & 2 & 22 & 9 \\
Freiburg & 30 & 5 & 20 & 5 \\
Frankfurt a.M. & 27 & 12 & 14 & 1 \\
Heidelberg & 15 & 2 & 10 & 3 \\
München & 17 & 0 & 16 & 1 \\
Tübingen & 33 & 9 & 19 & 5 \\
Insgesamt & 221 & 60 & 126 & 35 \\
\bottomrule()
\end{longtable}

\end{tablenos:no-prefix-table-caption}

Insgesamt handelte es sich bei über 96\% der Karzinome um mäßig (G2) und schlecht differenzierte bzw. undifferenzierte (G3) Primärtumoren im lokal fortgeschrittenen Tumorstadium pT3-T4 (TNM-Stadium) mit vorwiegend positivem Lymphknotenstatus (90\%, n=199). Zu Beginn der Strahlentherapie wurden bei keinem Patienten Fernmetastasen diagnostiziert. Diese traten erst im Laufe der Nachbeobachtung auf. Nach der UICC-Klassifikation befanden sich 8 (3,6\%) Patienten in Stadium II, 33 (14,9\%) in Stadium III und 180 (81,4\%) in Stadium IV. Die Stadien IV A und IV B wurden hierbei zum Stadium IV zusammengefasst. Das Stadium IV C entfällt aufgrund der Metastasenfreiheit aller Patienten zu Therapiebeginn. In

Tabelle \textbf{\emph{5‑3}} sind die Tumorcharakteristika des Patientenkollektivs im Detail zusammengefasst.

\textbf{Tabelle 5‑3: Tumorcharakteristik}

\begin{tablenos:no-prefix-table-caption}

\begin{longtable}[]{@{}lllll@{}}
\toprule()
Charakteristik & & n & & \% \\
\midrule()
\endhead
Tumorlokalisation & Mundhöhle & 60 & & 27.1 \\
& Oropharynx & 126 & & 57.0 \\
& Hypopharynx & 35 & & 15.8 \\
T-Stadium & pT2 & 8 & & 3.6 \\
& pT3 & 33 & & 14.9 \\
& pT4 & 180 & & 81.4 \\
N-Stadium & N0 & 22 & & 10.0 \\
& N+ & 199 & & 90.0 \\
Grading & 1 & 5 & & 2.3 \\
& 2 & 123 & & 55.7 \\
& 3 & 89 & & 40.3 \\
& 4 & 0 & & 0 \\
& Fehlend & 4 & & 1.8 \\
UICC-Stage & II & 8 & & 3.6 \\
& III & 33 & & 14.9 \\
& IV A, B & 180 & & 81.4 \\
R-Status & Negativ & 125 & & 56,6 \\
& Positiv & 94 & & 42.5 \\
& Fehlend & 2 & & 0.9 \\
ECE-Status & Negativ & 103 & & 46.6 \\
& Positiv & 118 & & 53.4 \\
HPV16 DNA (PCR) & Negativ & 143 & & 64.7 \\
& Positiv & 72 & & 32.6 \\
& Fehlend & 6 & & 2.7 \\
CD44-Expression (IHC) & Negativ & 44 & & 19.9 \\
& Positiv & 151 & & 68.3 \\
& Fehlend & 26 & & 11.8 \\
\bottomrule()
\end{longtable}

\end{tablenos:no-prefix-table-caption}

Die Untersuchung auf eine Infektion mit HPV wurde in einer vorrangegangenen Arbeit am gleichen Patientenkollektiv durchgeführt und ausführlich beschreiben (\href{about:blank\#_ENREF_52}{Lohaus et al., 2014}). Alle Tumorproben wurden dafür auf die Prävalenz von HPV16-DNA analysiert. Insgesamt litten 72 Patienten (33,5\%) an einen HPV16 DNA positiven Tumor, während 143 Patienten (64,7\%) HPV16 DNA-negative Tumoren aufwiesen. Es waren 6 Tumorpatienten (2,7\%) mit unbekanntem HPV-Status im Kollektiv eingeschlossen. Gemäß der internationalen Agentur für Krebsforschung (IARC) werden HPV16 DNA-positive Kopf-Hals-Plattenepithelkarzinome derzeit als HPV-assoziiert angesehen. Das bedeutet, dass eine persistierende HPV-Infektion als wahrscheinlichste Ursache der Tumorerkrankung anzunehmen ist (\href{about:blank\#_ENREF_15}{Bouvard et al., 2009}). Aus diesem Grund wurde der HPV-Status ausschließlich über den Nachweis von HPV16-DNA mittels PCR aus den vorhanden Tumormaterial ermittelt und für die weiteren Analysen verwendet.

Für die Immunhistochemische Analyse der CD44-Expression waren von den 221 eingeschlossenen Patienten Insgesamt 195 in Paraffin eingebettete Operationspräparate (FFPE) verfügbar. Von den analysierbaren Gewebeproben lag der Anteil der CD44-positiven Tumoren bei 77,4\% (n=151) und der Anteil CD44-negativer Tumoren 22,6\% (n=44). Bei 26 der 221 Patienten konnte der CD44-Status aufgrund des fehlenden bzw. ungenügenden Biomaterials nicht ermittelt werden. Die Ergebnisse der Biomarkeranalysen von HPV16-DNA und CD44 hinsichtlich deren Auftreten an den verschiedenen Tumorstellen sind in Tabelle 5‑4 dargestellt.

\textbf{Tabelle 5‑4: Anzahl der Tumoren mit positiven oder negativen Biomarkern Insgesamt und pro Tumorlokalisierung.} N=Anzahl der auswertbaren Fälle, F=Anzahl der nicht auswertbaren Fälle

\begin{tablenos:no-prefix-table-caption}

\begin{longtable}[]{@{}lllll@{}}
\toprule()
Biomarker & Insgesamt & Mundhöhle & Oropharynx & Hypopharynx \\
\midrule()
\endhead
HPV16 DNA & N=215, F=6 & N=58, F=2 & N=123, F=3 & N=34, F=1 \\
HPV16 DNA+ & 72 (33,5\%) & 7 (12,1\%) & 60 (48,0\%) & 5 (14,7\%) \\
HPV16 DNA- & 143 (66,5\%) & 51 (87,9\%) & 63 (51,2\%) & 29 (85,3\%) \\
CD44 & N=195, F=26 & N=54, F=6 & N=113, F= 13 & N=28, F=7 \\
CD44+ & 151 (77,4\%) & 51 (94,4\%) & 78 (69,0\%) & 22 (78,6\%) \\
CD44 - & 44 (22,6\%) & 3 (5,6\%) & 35 (30,1\%) & 6 (21,4\%) \\
\bottomrule()
\end{longtable}

\end{tablenos:no-prefix-table-caption}

Bei den Patienten der Subgruppe mit HPV16 DNA-negativen Kopf-Hals-Plattenepithel-karzinomen, lag der Anteil CD44-positiver Karzinome mit 116 Fälle bei .der Subgruppe hatten ein CD44-negativen Tumor.

\hypertarget{behandlungsmerkmale}{%
\subsubsection{Behandlungsmerkmale}\label{behandlungsmerkmale}}

\emph{Chirurgie}

Präoperativ erhielten alle Patienten gemäß der S3-Leitlinie (\href{about:blank\#_ENREF_14}{Bootz, 2020}) die benötigten Staginguntersuchungen. Aufgrund des lokal fortgeschrittenen Krankheitsgeschehens war bei allen Patienten noch vor Beginn der adjuvanten Radiochemotherapie die operative Entfernung des Tumors indiziert. Dabei kamen klassische chirurgische Therapieoptionen als auch transoral-endoskopische Laserchirurgie zur Anwendung. Die radikale Entfernung des Tumors (En-bloc-Resektion) sowie der ableitenden Lymphbahnen erhielten dabei die höchste Priorität. Dafür kamen in den DKTK-Partner-Kliniken die unterschiedlichen Varianten der Neck Dissection und, bei ausgedehnten Hart- und Weichteildefekten, plastischen Rekonstruktionen zum Einsatz. Ein extrakapsuläres Tumorwachstum war bei rund die Hälfte (53,4\%, n=118) der Tumorpatienten festgestellt worden. Eine tumorfreier Resektionsrand (R0- Resektion) konnte in 56,6\% (n=125) der Fälle erzielt werden. Bei 42,5\% (n=94) der Fälle zeigte sich mindestens ein mikroskopischer Resektionsrand (R1/R2). Bei zwei Patienten (0,9\%) konnte das Vorhandensein eines Residualtumors nicht eindeutig beurteilt werden (Rx). Die histologischen Präparate wurden an den DKTK-Partnerstandorten standardisiert aufgearbeitet und gemäß der TMN-Klassifikation eingeordnet.

\emph{Bestrahlung und Chemotherapie}

Die Behandlungsparameter bezüglich der Chemo- und Strahlentherapie sind in Tabelle 5‑5 übersichtlich zusammengefasst. Zwischen der Operation und dem Beginn der Strahlentherapie verging eine mediane Zeit von 6 Wochen mit einer Spannweite von 1 bis 23 Wochen. Bei 166 Patienten (75,1\%) vergingen nach der Operation bis zu 49 Tage bis zum Beginn der Strahlentherapie, bei 40 Patienten (18,1\%) vergingen mehr als 49 Tage. Bei 15 Patienten (6,8\%) war diesbezüglich keine Daten vorhanden. Die mediane Dauer der strahlentherapeutischen Behandlungszeit betrug 44 Tage und reichte von 32 bis zu 57 Tagen. Die konventionelle Technik der CT-gestützten dreidimensionalen Bestrahlungsplanung (3D-RT) wurde bei allen Patienten angewendet. Die Behandlung wurde in der üblichen Shrinkingfield-Technik durchgeführt, wobei zunächst die Tumorregionen, die befallenen Lymphknotenstationen und die elektiven, regionären Lymphabflussgebiete mit Einzeldosen von 1,8 -- 2,1 Gy für fünf- bis sechsmal wöchentlich, bis zu einer medianen Gesamtdosis von 50,4 Gy (Range: 46,8 Gy -- 66,0 Gy) behandelt wurden. Anschließend erfolgte eine simultane integrierte Boostbestrahlung der primären Tumorregion sowie der befallenen Lymphknoten Stationen entsprechend dem Risikoprofil bis zu einer medianen kumulativen Gesamtdosis von 64,0 Gy mit einer Spannweite zwischen 56,0 Gy und 68,4 Gy.

\textbf{Tabelle 5‑5: Behandlungsparameter}

\begin{tablenos:no-prefix-table-caption}

\begin{longtable}[]{@{}
  >{\raggedright\arraybackslash}p{(\columnwidth - 12\tabcolsep) * \real{0.1429}}
  >{\raggedright\arraybackslash}p{(\columnwidth - 12\tabcolsep) * \real{0.1429}}
  >{\raggedright\arraybackslash}p{(\columnwidth - 12\tabcolsep) * \real{0.1429}}
  >{\raggedright\arraybackslash}p{(\columnwidth - 12\tabcolsep) * \real{0.1429}}
  >{\raggedright\arraybackslash}p{(\columnwidth - 12\tabcolsep) * \real{0.1429}}
  >{\raggedright\arraybackslash}p{(\columnwidth - 12\tabcolsep) * \real{0.1429}}
  >{\raggedright\arraybackslash}p{(\columnwidth - 12\tabcolsep) * \real{0.1429}}@{}}
\toprule()
\begin{minipage}[b]{\linewidth}\raggedright
Behandlungsparameter
\end{minipage} & \begin{minipage}[b]{\linewidth}\raggedright
Median
\end{minipage} & \begin{minipage}[b]{\linewidth}\raggedright
Perzentile
\end{minipage} & \begin{minipage}[b]{\linewidth}\raggedright
\end{minipage} & \begin{minipage}[b]{\linewidth}\raggedright
\end{minipage} & \begin{minipage}[b]{\linewidth}\raggedright
\end{minipage} & \begin{minipage}[b]{\linewidth}\raggedright
Rang
\end{minipage} \\
\midrule()
\endhead
& & 10\% & 25\% & 75\% & 90\% & \\
Cisplatin-Dosis (mg/m2 KOF) & 200 & 100 & 200 & 200 & 240 & 100-300 \\
RT-Dosis (Gy) & & & & & & \\
Boost-Volumen & 64.0 & 60.0 & 63.9 & 66.0 & 66.0 & 56.0 - 68.4 \\
Pro Fraktion & 2.0 & 1.8 & 1.8 & 2.0 & 2.0 & 1.8 - 2.1 \\
Adjuvantes Volumen & 50.4 & 50.0 & 50.0 & 55.9 & 60.0 & 46.8 -- 66.0 \\
Pro Fraktion & 2.0 & 1.8 & 2.0 & 2.0 & 2.1 & 1.8 -- 2.2 \\
Zeit zwischen OP und Bestrahlung (Wochen) & 6.0 & 4.0 & 5.0 & 8.0 & 10.0 & 1.0 -- 23.0 \\
Gesamtbehandlungszeit der PORT-C (Tage) & 44.0 & 41.0 & 43.0 & 47.0 & 50.8 & 32.0 -- 57.0 \\
Follow-up / Nachbeobachtung (Monate) & 47.3 & 0.8 & 2.6 & 5.2 & 6.1 & 0.2 -- 8.3 \\
\bottomrule()
\end{longtable}

\end{tablenos:no-prefix-table-caption}

\emph{Simultane Chemotherapie}

Eine zur Radiotherapie simultane Chemotherapie wurde bei allen Patienten durchgeführt. Die Kombination aus Cisplatin und 5-Fluoruracil wurde 70,8\% (n=75) der Patienten und somit bevorzugt verabreicht. 10,4\% (n=11) erhielten eine Monotherapie mit Mitomycin C, jeweils 6,6\% (n=7) die Kombination aus Mitomycin und 5-Fluoruracil oder die Dreifachkombination aus Cisplatin, Cetuximab und 5-Fluoruracil, letztere gemäß der ACCRA-Studie, und schließlich erhielten jeweils 2,8 \% (n=3) Patienten die Monotherapie bestehend aus Cisplatin (40mg/m2 Körperoberfläche wöchentlich) oder aus Cetuximab.

\textbf{Patienten- und Tumorcharakteristika in Abhängigkeit des HPV16 DNA- und CD44 Status}

Lediglich 17 Patienten (7,7\%) hatten zu Beginn der Strahlentherapie niemals geraucht. 20 (9,0\%) waren ehemals Raucher und knapp die Hälfte (48,4\%; n=107) rauchte während der Behandlungszeit. Somit konnte bei einem Großteil der Patienten (57,4\%) regelmäßiger Tabakgebrauch gegenwärtig der Behandlung oder in der Vergangenheit verzeichnet werden. Niemals Alkohol getrunken zu haben gaben lediglich 30 Patienten (13,6\%) an. 27 Patienten (12,2\%) haben in der Vergangenheit regelmäßig Alkohol getrunken und waren zu Beginn des Beobachtungszeitraums mindestens 1 Jahr Alkoholabstinent. Aussagen zum regelmäßigen Alkoholkonsum vor und während des Beobachtungszeitraum konnten nicht erhoben werden.

\begin{tablenos:no-prefix-table-caption}

\begin{longtable}[]{@{}llllll@{}}
\toprule()
& Insgesamt & LRC & DM & OS & Zensiert \\
\midrule()
\endhead
Alle Patienten & N= 221 & N=29 & N=42 & N=70 & N=80 \\
Mundhöhle & 60 & 15 & 18 & 28 & -1 \\
Oropharynx & 126 & 11 & 14 & 34 & 67 \\
Hypopharynx & 35 & 3 & 10 & 8 & 14 \\
\bottomrule()
\end{longtable}

\end{tablenos:no-prefix-table-caption}

\hypertarget{ereigniszeitanalysen}{%
\subsection{Ereigniszeitanalysen}\label{ereigniszeitanalysen}}

Um die prognostische Relevanz von CD44 als Tumorstammzellmarker für lokal fortgeschrittene Kopf-Hals Plattenepithelkarzinome bei postoperativer Radiochemotherapie zu überprüfen, wurde in dieser Arbeit die CD44-Proteinexpression im prätherapeutischen Tumorgewebematerial mit verschiedenen klinischen Endpunkten korreliert. Darüber hinaus wurde das klinische Outcome unter Einbeziehung weiterer klinisch-pathologischer Parameter mittels uni- und multivariater Analysen berechnet. Primärer Endpunkt war die lokoregionäre Tumorkontrolle (LRC), sekundäre Endpunkte waren Fernmetastasen-freie Überleben (DMS) und Gesamtüberleben (OS).

\hypertarget{deskription-der-ereigniszeitdaten-im-beobachtungszeitraum}{%
\subsubsection{Deskription der Ereigniszeitdaten im Beobachtungszeitraum}\label{deskription-der-ereigniszeitdaten-im-beobachtungszeitraum}}

Die Anzahl aller Ereignisse für die loko-regionäre Tumorkontrolle (lokales Versagen), das Fernmetastasen-freies Überleben (entferntes Versagen) und für das Gesamtüberleben (Tod), die im Beobachtungszeitraum sowohl im gesamten Patientenkollektiv als auch in der Subgruppe der HPV16 DNA-positiven und -negativen Kohorte erfasst wurden, sind in Tabelle 5‑6 und in Anlage , entsprechend der Tumorlokalisation dargestellt.

\textbf{Tabelle 5‑6: Anzahl der Ereignisse für die klinischen Endpunkte im Beobachtungszeitraum pro Tumorlokalisation vom gesamten Patientenkollektiv.} N=Anzahl der auswertbaren Fälle

\begin{tablenos:no-prefix-table-caption}

\begin{longtable}[]{@{}
  >{\raggedright\arraybackslash}p{(\columnwidth - 8\tabcolsep) * \real{0.2000}}
  >{\raggedright\arraybackslash}p{(\columnwidth - 8\tabcolsep) * \real{0.2000}}
  >{\raggedright\arraybackslash}p{(\columnwidth - 8\tabcolsep) * \real{0.2000}}
  >{\raggedright\arraybackslash}p{(\columnwidth - 8\tabcolsep) * \real{0.2000}}
  >{\raggedright\arraybackslash}p{(\columnwidth - 8\tabcolsep) * \real{0.2000}}@{}}
\toprule()
\begin{minipage}[b]{\linewidth}\raggedright
Endpunkte
\end{minipage} & \begin{minipage}[b]{\linewidth}\raggedright
Insgesamt
\end{minipage} & \begin{minipage}[b]{\linewidth}\raggedright
Mundhöhle
\end{minipage} & \begin{minipage}[b]{\linewidth}\raggedright
Oropharynx
\end{minipage} & \begin{minipage}[b]{\linewidth}\raggedright
Hypopharynx
\end{minipage} \\
\midrule()
\endhead
Alle Patienten & N= 221 & N=60 & N=126 & N=35 \\
Lokales Versagen & 29 (13,1\%) & 15 (25,0\%) & 11 (8,7\%) & 3 (8,6\%) \\
Entferntes Versagen & 42 (19,0\%) & 18 (30,0\%) & 14 (11,1\%) & 10 (28,6\%) \\
Tod & 70 (31,7\%) & 28 (46,7\%) & 34 (27,0\%) & 8 (22,9\%) \\
Zensiert & 80 (36,2\%) & 0 (0,0\%) & 66 (52,4\%) & 14 (40\%) \\
\bottomrule()
\end{longtable}

\end{tablenos:no-prefix-table-caption}

Die medianen Nachbeobachtungszeiten des Patientenkollektivs ab Beginn der Strahlentherapie bis zum Erreichen der jeweiligen Endpunkte oder Zensierungen betrugen im durschnitt 47,1 Monate (3,9 Jahre) mit einer mittleren Spannweite von 2,2 bis 100,1 Monaten. In diesem Zeitraum kam es bei 29 der 221 (13,1\%) beobachteten Patienten zu einem loko-regionären Rezidiv im Bereich des Bestrahlungsvolumens. Das erste Rezidiv wurde nach 2,1 Monaten im Bereich der Mundhöhle diagnostiziert. Das längste Intervall bis zum Auftreten eines loko-regionären Rezidivs betrug 68,2 Monate und trat im Bereich des Hypopharynx auf. Anteilsmäßig traten die meisten Rezidive mit 25\% im Bereich der Mundhöhle auf. Lediglich bei 8,7\% der Patienten mit Oropharynxkarzinom und bei 8,6\% der Patienten mit Hypopharynxkarzinom wurde im Beobachtungszeitraum ein loko-regionäres Rezidiv diagnostiziert. Die nach Kaplan-Meier geschätzten Ereignisraten der loko-regionären Tumorkontrolle sind in Abbildung 5‑1 dargestellt. Postoperativ waren nach 2 Jahren 89,6 \% (Standardfehler 2,1\%) der Patienten lokal und regional Rezidiv-frei und nach 5 Jahren 85,7\% (Standardfehler 2,6\%).Da im Beobachtungszeitraum in allen untersuchten Endpunkten bei weniger als 50\% der Klienten ein Ereignis eingetreten war, konnten die medianen Überlebenszeiten nicht berechnet werden. Die mediane Ereignisfreie Zeit beträgt daher mindestens 60 Monate. Die Ereignis-freie Zeit bezogen auf die loko-regionäre Tumorkontrolle betrug durchschnittlich 87,3 Monate (95\%-KI, 82,8 bis 91,7 Monate).\href{about:blank\#fn1}{1}

\begin{fignos:no-prefix-figure-caption}

\begin{figure}
\centering
\includegraphics{Abbildungen/image8.png}
\caption{Abbildungen/image8.png}
\end{figure}

\end{fignos:no-prefix-figure-caption}

\textbf{Abbildung 5‑1: Lokoregionäre Kontrollrate nach Kaplan-Meier für das Gesamtkollektiv}

Fernmetastasen traten bei 42 Patienten (19,0\% der Fälle) auf. Die Abbildung 5‑2 zeigt die Kaplan-Meier Ereigniskurve für das Fernmetastasen-freie Überleben des gesamten Patientkollektivs. Die fernmetastasenfreie Ereignisrate nach 2 und 5 Jahren betrug 85,1\% (Standardfehler 2,4\%) sowie 79,1\% (Standardfehler 2,9\%). Die mittlere ereignisfreie Zeit lag bei 82,8 Monaten (95\%-KI, 78,1 bis 87,5 Monate).

\begin{quote}
\end{quote}

\textbf{Abbildung 5‑2: Fernmetastasenfreies Überleben nach Kaplan-Meier für das Gesamtkollektiv}

Zum Ende der Beobachtungszeit waren 70 Patienten (31,7 \%) verstorben. Die 2- bzw. 5-Jahresüberlebensrate betrug schätzungsweise 83,9\% (Standardfehler 2,5\%) bzw. 67,8\% (Standardfehler 3,4\%). Die entsprechende Kaplan-Meier Kurve zeigt Abbildung 5‑3. Die mittlere Überlebenszeit lag bei 71,6 Monaten (95\%-KI, 65,8 bis 72,3 Monate).

\begin{quote}
\end{quote}

\textbf{Abbildung 5‑3: Gesamtüberleben nach Kaplan-Meier für das Gesamtkollektiv.}

Im Untersuchungszeitraum verstarben 70 Patienten, bei 29 wurde ein lokoregionäres Rezidiv diagnostiziert und 42 Patienten bekamen Fernmetastasen. Insgesamt wurden also 141 Ereignisse registriert. Somit war für 80 (36,2\%) Patienten das Follow-up zum Zeitpunkt der letzten Datenerfassung der vorliegenden Arbeit noch nicht abgeschlossen. Diese wurden als zensierte Ereignisse in der Datenauswertung berücksichtigt. Die Kaplan-Meier Ereigniskurven für das Patientenkollektiv der Subgruppen mit HPV16 DNA-negativen und -positiven Tumoren, können der Anlage entnommen werden.

\hypertarget{univariate-analysen-zur-prognostische-relevanz-von-cd44}{%
\subsubsection{Univariate Analysen zur prognostische Relevanz von CD44}\label{univariate-analysen-zur-prognostische-relevanz-von-cd44}}

Der potenzielle Tumorstammzellmarker (CSC-Marker) CD44 wurde anhand seiner immunhistochemische Proteinexpression im Tumormaterial des Patientkollektivs, wie oben beschreiben bewertet. Je nach Expressionsverhalten, konnte das Gesamtkollektiv dadurch in CD44-positive und -negative Gruppen stratifiziert werden. 151 Patienten wiesen Tumore mit CD44-positiver Proteinexpression auf, während 44 Patienten keine Färbeaktivität auf den Antikörper gegen CD44 zeigten und als CD44-negativ definiert wurden. Tabelle 5‑7 zeigt die Verteilung der CD44 positiven und negativen Tumoren auf die klinischen Ereignisse bzw. Zensierungen im Beobachtungszeitraum sowohl vom gesamten Patientenkollektiv als auch von den beiden Subgruppen mit HPV16 DNA-positiven und negativen Tumoren.

\textbf{Tabelle 5‑7: Anzahl der Tumoren mit positiver oder negativer CD44 Proteinexpression pro Ereignis bzw. Zensierung.} N=Anzahl der auswertbaren Fälle in der Kohorte

\begin{tablenos:no-prefix-table-caption}

\begin{longtable}[]{@{}
  >{\raggedright\arraybackslash}p{(\columnwidth - 10\tabcolsep) * \real{0.1667}}
  >{\raggedright\arraybackslash}p{(\columnwidth - 10\tabcolsep) * \real{0.1667}}
  >{\raggedright\arraybackslash}p{(\columnwidth - 10\tabcolsep) * \real{0.1667}}
  >{\raggedright\arraybackslash}p{(\columnwidth - 10\tabcolsep) * \real{0.1667}}
  >{\raggedright\arraybackslash}p{(\columnwidth - 10\tabcolsep) * \real{0.1667}}
  >{\raggedright\arraybackslash}p{(\columnwidth - 10\tabcolsep) * \real{0.1667}}@{}}
\toprule()
\begin{minipage}[b]{\linewidth}\raggedright
\end{minipage} & \begin{minipage}[b]{\linewidth}\raggedright
Gesamt
\end{minipage} & \begin{minipage}[b]{\linewidth}\raggedright
Rezidiv
\end{minipage} & \begin{minipage}[b]{\linewidth}\raggedright
Fernmetastasen
\end{minipage} & \begin{minipage}[b]{\linewidth}\raggedright
Tod
\end{minipage} & \begin{minipage}[b]{\linewidth}\raggedright
Zensiert
\end{minipage} \\
\midrule()
\endhead
Alle Patienten & N=221 & N=29 & N=42 & N=70 & N=80 \\
CD44 + & 151 (68,3\%) & 28 (96,6\%) & 34 (81,0\%) & 56 (80,0\%) & 33 (41,3\%) \\
CD44 - & 44 (19,9\%) & 1 (3,4\%) & 5 (11,9\%) & 10 (14,3\%) & 28 (35,0\%) \\
Fehlend & 26 (11,8\%) & 0 (0,0\%) & 3 (7,1\%) & 4 (5,7\%) & 19 (23,7\%) \\
HPV16 DNA- & N=143 & N=27 & N=34 & N=57 & N=25 \\
CD44 + & 116 (81,1\%) & 27 (100,0\%) & 31 (91,2\%) & 51 (89,5\%) & 7 (28,0\%) \\
CD44 - & 14 (9,8\%) & 0 (0,0\%) & 2 (5,9\%) & 5 (8.8\%) & 7 (28,0\%) \\
Fehlend & 13 (9,1\%) & 0 (0,0\%) & 1 (2,9\%) & 1 (1,7\%) & 11 (4,0\%) \\
HPV16 DNA+ & N=72 & N=2 & N=7 & N=11 & N=52 \\
CD44 + & 35 (48,6\%) & 1 (50,0\%) & 3 (42,9\%) & 5 (45,5\%) & 26 (50,0\%) \\
CD44 - & 30 (41,7\%) & 1 (50,0\%) & 3 (43,9\%) & 5 (45,5\%) & 21 (40.4\%) \\
Fehlend & 7 (9,7\%) & 0 (0,0\%) & 1 (14,2\%) & 1 (9,1\%) & 5 (9.6\%) \\
\bottomrule()
\end{longtable}

\end{tablenos:no-prefix-table-caption}

Um herauszufinden, ob das Vorhandensein einer CD44-Proteinexpression im Tumorgewebe mit dem klinischen Outcome korreliert, wurden in Abbildung 5‑4 die Ereigniskurven der CD44-positiven und -negativen Patientenkollektive für die loko-regionäre Tumorkontrolle skizziert. Die entsprechenden Überlebenskurven für die sekundären Endpunkte sind in Abbildung 5‑6 dargestellt. Mit Hilfe des Log-Rank Test wurde anschließend jeweils überprüft, ob zwischen den CD44+ und CD44- Ereigniskurven ein signifikanter Unterscheid auf einem Signifikanzniveau von 5\% besteht. Die univariaten Analysen zeigen eine signifikante Assoziation der CD44-Proteinexpression mit der loko-regionären Tumorkontrolle (HR 9,09, p = 0,008; siehe Abbildung 5‑4) aber nicht mit den sekundären Endpunkten Fernmetasten-freies Überleben (HR 2,29, p = 0,075) oder Gesamtüberleben (HR 1,78, p = 0,089). Die dazugehörigen Kaplan-Meier Ereigniskurven für die sekundären Endpunkte Fernmetastasen-freies Überleben und Gesamtüberleben sind weiter unten in Abbildung 5‑6 dargestellt.

\begin{quote}
\end{quote}

\textbf{Abbildung 5‑4: Kaplan-Meier Analyse zum Einfluss der CD44 Proteinexpression auf die loko-regionäre Tumorkontrolle aller Patienten.} Patienten mit CD44 positiven Kopf-Hals-Plattenepithelkarzinomen haben eine signifikant schlechtere loko-regionäre Tumorkontrolle als Patienten mit CD44 negativen Tumoren (p=0,008)

Ähnliche Effekte für die loko-regionäre Kontrolle wurden in der Untergruppe der HPV16 DNA-negativen Kopf-Hals-Plattenepithelkarzinomen beobachtet. Hier zeigten univariate Analysen einen statistischen Trend für eine Korrelation von CD44 mit der loko-regionären Tumorkontrolle (p = 0,05; Abbildung 5‑5).

\begin{quote}
\end{quote}

\textbf{Abbildung 5‑5: Kaplan-Meier Analyse zum Einfluss der CD44 Proteinexpression auf die loko-regionäre Tumorkontrolle von Patienten mit HPV16 DNA-negativen Tumoren.} Patienten der Subgruppe mit HPV16 DNA-negativen und CD44 negativen Kopf-Hals-Plattenepithelkarzinomen zeigen einen statistischen Trend für eine bessere loko-regionäre Tumorkontrolle im Vergleich zu Patienten mit HPV16 DNA negativen und CD44-positiven Tumoren (p=0,05).

Ähnlich wie bei dem gesamten Patientenkollektiv, hatte CD44 jedoch keinen Einfluss auf die sekundären Endpunkte von Patienten der Subgruppe mit HPV16 DNA-negativen Tumoren. Abbildung 5‑6 zeigt die dazugehörigen Überlebensfunktion dieser Subgruppe. Alle Ergebnisse der univariaten Analyse (Log-Rank Test) zum Einfluss von CD44 auf die loko-regionäre Kontrolle, Fernmetastasen-freies Überleben und Gesamtüberleben der gesamten Patientenkohorte sowie der Subgruppe der HPV16 DNA-negativen Kohorte sind in Tabelle 5‑8 nochmal zusammengefasst.

\textbf{Tabelle 5‑8: Univariate Analysen zum Einfluss von CD44 als Tumorstammzellmarker auf die loko-regionären Kontrolle, Fernmetastasen-freies Überleben und Gesamtüberleben.} HR = Hazard Ratio; 95\% KI = 95\% -Konfidenzintervall

\begin{tablenos:no-prefix-table-caption}

\begin{longtable}[]{@{}
  >{\raggedright\arraybackslash}p{(\columnwidth - 14\tabcolsep) * \real{0.1250}}
  >{\raggedright\arraybackslash}p{(\columnwidth - 14\tabcolsep) * \real{0.1250}}
  >{\raggedright\arraybackslash}p{(\columnwidth - 14\tabcolsep) * \real{0.1250}}
  >{\raggedright\arraybackslash}p{(\columnwidth - 14\tabcolsep) * \real{0.1250}}
  >{\raggedright\arraybackslash}p{(\columnwidth - 14\tabcolsep) * \real{0.1250}}
  >{\raggedright\arraybackslash}p{(\columnwidth - 14\tabcolsep) * \real{0.1250}}
  >{\raggedright\arraybackslash}p{(\columnwidth - 14\tabcolsep) * \real{0.1250}}
  >{\raggedright\arraybackslash}p{(\columnwidth - 14\tabcolsep) * \real{0.1250}}@{}}
\toprule()
\begin{minipage}[b]{\linewidth}\raggedright
Kohorte
\end{minipage} & \begin{minipage}[b]{\linewidth}\raggedright
\end{minipage} & \begin{minipage}[b]{\linewidth}\raggedright
Loko-regionäre Kontrolle
\end{minipage} & \begin{minipage}[b]{\linewidth}\raggedright
\end{minipage} & \begin{minipage}[b]{\linewidth}\raggedright
Fernmetastasen-freies Überleben
\end{minipage} & \begin{minipage}[b]{\linewidth}\raggedright
\end{minipage} & \begin{minipage}[b]{\linewidth}\raggedright
Gesamtüberleben
\end{minipage} & \begin{minipage}[b]{\linewidth}\raggedright
\end{minipage} \\
\midrule()
\endhead
& & HR (95\% KI) & p-Wert & HR (95\% KI) & p-Wert & HR (95\% KI) & p-Wert \\
Alle Patienten & & 9,09 (1,24-66,8) & 0,008 & 2,29 (0,90-5,86) & 0,075 & 1,78 (0,91-3,50) & 0,089 \\
HPV16 DNA neg & & 5,21 (0,69-39,45) & 0,053 & 1,49 (0,56-3,99) & 0,421 & 1,16 (0,57-2,34) & 0,680 \\
\bottomrule()
\end{longtable}

\end{tablenos:no-prefix-table-caption}

\begin{tablenos:no-prefix-table-caption}

\begin{longtable}[]{@{}ll@{}}
\toprule()
Alle Patienten & Patienten mit HPV16 DNA-negativen jjjjjjjjjTumoren \\
\midrule()
\endhead
& \\
\bottomrule()
\end{longtable}

\end{tablenos:no-prefix-table-caption}

\begin{fignos:no-prefix-figure-caption}

\begin{figure}
\centering
\includegraphics{Abbildungen/image13.png}
\caption{Abbildungen/image13.png}
\end{figure}

\end{fignos:no-prefix-figure-caption}

\begin{fignos:no-prefix-figure-caption}

\begin{figure}
\centering
\includegraphics{Abbildungen/image14.png}
\caption{Abbildungen/image14.png}
\end{figure}

\end{fignos:no-prefix-figure-caption}

\begin{fignos:no-prefix-figure-caption}

\begin{figure}
\centering
\includegraphics{Abbildungen/image15.png}
\caption{Abbildungen/image15.png}
\end{figure}

\end{fignos:no-prefix-figure-caption}

\begin{fignos:no-prefix-figure-caption}

\begin{figure}
\centering
\includegraphics{Abbildungen/image16.png}
\caption{Abbildungen/image16.png}
\end{figure}

\end{fignos:no-prefix-figure-caption}

\textbf{Abbildung 5‑6: Kaplan-Meier Schätzungen der sekundären Endpunkte von CD44-Protein (bewertet durch IHC) allen Patienten oder (C, E, G, I) Patienten mit HPV16 DNA-negativem HNSCC.} Patienten mit geringer oder keiner CSC-Markerexpression zeigten eine bessere loko-regionale Tumorkontrolle im Vergleich zu Patienten mit stärkerer CSC-Markerexpression.

\begin{fignos:no-prefix-figure-caption}

\begin{figure}
\centering
\includegraphics{Abbildungen/image17.png}
\caption{Abbildungen/image17.png}
\end{figure}

\end{fignos:no-prefix-figure-caption}

In univariate Analysen wurden das Patientenkollektiv stratifiziert in CD44 positive und negative und die Überlebensdaten der beider Gruppen mit Hilfe des Kaplane-Meier Verfahrens untersucht. Die geschätzten Überlebenszeiten beide Gruppen wurden m.H. des Log-rank Test miteinander verglichen. Die Untersuchung ergab eine ein signifikant besseres klinisches outcome einen hochsignifikanten Überlebensvorteil für Patienten mit p16INK4A- positiven Karzinomen (p \textless{} 0,001) (Abb.16A)

\hypertarget{univariate-analysen-zur-identifizierung-weitere-prognostische-parameter}{%
\subsubsection{Univariate Analysen zur Identifizierung weitere prognostische Parameter}\label{univariate-analysen-zur-identifizierung-weitere-prognostische-parameter}}

Zur Identifizierung potenzielle Prognosefaktoren für Patienten mit lokal fortgeschritten Kopf-Hals-Plattenepithelkarzinomen nach postoperativer Radiochemotherapie wurden im Rahmen univariater Analysen verschiedene klinisch-pathologische Parameter mit dem klinischen Outcome korreliert. Die Ergebnisse sind in Tabelle 5‑9 zusammengefasst.

\textbf{Tabelle 5‑9: Univariate Analyse potenzieller prognostische Parameter für die loko-regionäre Tumorkontrolle, Fernmetastasen-freies Überleben und Gesamtüberleben.} HR = Hazard Ratio; 95\% KI = 95\% Konfidenzintervall

\begin{tablenos:no-prefix-table-caption}

\begin{longtable}[]{@{}
  >{\raggedright\arraybackslash}p{(\columnwidth - 12\tabcolsep) * \real{0.1429}}
  >{\raggedright\arraybackslash}p{(\columnwidth - 12\tabcolsep) * \real{0.1429}}
  >{\raggedright\arraybackslash}p{(\columnwidth - 12\tabcolsep) * \real{0.1429}}
  >{\raggedright\arraybackslash}p{(\columnwidth - 12\tabcolsep) * \real{0.1429}}
  >{\raggedright\arraybackslash}p{(\columnwidth - 12\tabcolsep) * \real{0.1429}}
  >{\raggedright\arraybackslash}p{(\columnwidth - 12\tabcolsep) * \real{0.1429}}
  >{\raggedright\arraybackslash}p{(\columnwidth - 12\tabcolsep) * \real{0.1429}}@{}}
\toprule()
\begin{minipage}[b]{\linewidth}\raggedright
Parameter
\end{minipage} & \begin{minipage}[b]{\linewidth}\raggedright
Loko-regionäre Kontrolle
\end{minipage} & \begin{minipage}[b]{\linewidth}\raggedright
\end{minipage} & \begin{minipage}[b]{\linewidth}\raggedright
Fernmetastasen-freies Überleben
\end{minipage} & \begin{minipage}[b]{\linewidth}\raggedright
\end{minipage} & \begin{minipage}[b]{\linewidth}\raggedright
Gesamtüberleben
\end{minipage} & \begin{minipage}[b]{\linewidth}\raggedright
\end{minipage} \\
\midrule()
\endhead
& HR (95\% KI) & p-Wert & HR (95\% KI) & p-Wert & HR (95\% KI) & p-Wert \\
Tumorlokalisation & & & & & & \\
Mundhöhle* & 3,86 (1,85-8,04) & \textless0,01 & 2,58 (1,39-4,76) & \textless0,01 & 2,59 (1,60-4,19) & \textless0,01 \\
Oropharynx* & 0,38 (0,18-0,82) & 0,01 & 0,30 (0,16-0,58) & \textless0,01 & 0,58 (0,36-0,93) & 0,01 \\
Hypopharynx* & 0,57 (0,17-1,90) & 0,36 & 1,75 (0,86-3,55) & 0,12 & 0,60 (0,29-1,26) & 0,21 \\
Geschlecht & & & & & & \\
Mann vs.~Frau & 1,79 (0,79-4,05) & 0,16 & 0,59 (0,23-1,50) & 0,27 & 1,24 (0,70-2,20) & 0,45 \\
UICC-Status & & & & & & \\
II* & 0,96 (0,13-7,04) & 0,97 & 0,67 (0,09-4,87) & 0,69 & 1,06 (0,33-3,39) & 0,92 \\
III* & 0,91 (0,32-2,61) & 0,86 & 0,75 (0,30-1,91) & 0,55 & 1,15 (0,62-2,14) & 0,66 \\
IV* & 1,10 (0,42-2,88) & 0,85 & 1,39 (0,59-3,29) & 0,46 & 0,88 (0,50-1,56) & 0,65 \\
R Status & 1,08 (0,52-2,26) & 0,83 & 0,98 (0,53-1,82) & 0,95 & 1,10 (0,69-1,76) & 0,69 \\
ECE Status & 1,57 (0,74-3,34) & 0,24 & 2,68 (1,35-5.33) & \textless0,01 & 1,72 (1,06-2,80) & 0,03 \\
HPV16 DNA & 0,13 (0,03-0,54) & \textless0,01 & 0,35 (0,15-0,79) & 0,01 & 0,33 (0,17-0,62) & \textless0,01 \\
* vs.~others & & & & & & \\
\bottomrule()
\end{longtable}

\end{tablenos:no-prefix-table-caption}

Die univariate Analyse ermittelte in Bezug auf die Tumorlokalisation in der Mundhöhle (LRC: HR 3,68, p\textless0,01) und Oropharynx (LRC: HR 0,38, p\textless0,01) signifikante Unterschiede in allen drei Endpunkten. Wohingegen der ECE-Status lediglich erhebliche Auswirkungen auf die sekundären Endpunkte (DMS: HR 2,68, p\textless0,01; OS: HR 1,72, p=0,03) gezeigt hatte. Für das Geschlecht, dem UICC-Stadium und dem R-Status konnten keine signifikanten Auswirkungen auf das klinische Outcome festgestellt werden. HPV16 DNA-positive Tumoren zeigten über alle Endpunkte ein signifikant besseres klinisches Outcome als HPV16 DNA-negative Tumore.

\hypertarget{multivariate-analysen}{%
\subsubsection{Multivariate Analysen}\label{multivariate-analysen}}

Um zu untersuchen, ob die Proteinexpression von CD44 einen prognostischen Tumorstammzellmarker unabhängig von anderen klinischen Parametern darstellt, wurde eine multivariate Analyse unter Berücksichtigung andere Einflussgrößen durchgeführt. Parameter, die in der univariaten Analyse als signifikant befunden wurden, wurden als Einflussgrößen in ein multivariates Cox- (Proportional-Hazard) Regressionsmodel aufgenommen. Als Kovariaten für dieses Cox-Modell wurden demnach neben dem CD44-Status der ECE-Status, die Tumorlokalisation in Mundhöhle und Oropharynx sowie der HPV16 DNA-Status als gemeinsame gleichzeitige Einflussvariablen auf die Überlebenszeit, Ereigniszeit berücksichtigt. Ein signifikanter P-Wert in der Cox-Regression bedeutet, dass die Kovariate Einfluss auf das Überleben, die Ereigniszeit hat. Die Cox-Regression bietet außerdem die Möglichkeit, einen Schätzer für die Größe des Einflusses zu erhalten. Dieser Schätzer ist durch das Hazard Ratio (HR) gegeben und wurde für jede Kovariate berechnet. = Risiko für das Eintreten eines Ereignisses bei dieser Kovariate. Die verschiedenen Interpretationsmöglichkeiten des HR sind in Tabelle angegeben.

\textbf{Tabelle 5‑10: Hazard Ratio ist ein deskriptives Maß für das Einflussrisiko im Verhältnis zur Referenzgruppe}

\begin{tablenos:no-prefix-table-caption}

\begin{longtable}[]{@{}
  >{\raggedright\arraybackslash}p{(\columnwidth - 2\tabcolsep) * \real{0.5000}}
  >{\raggedright\arraybackslash}p{(\columnwidth - 2\tabcolsep) * \real{0.5000}}@{}}
\toprule()
\begin{minipage}[b]{\linewidth}\raggedright
HR = 0
\end{minipage} & \begin{minipage}[b]{\linewidth}\raggedright
Ereignisrisiko in beiden Gruppen ungefähr gleich
\end{minipage} \\
\midrule()
\endhead
HR \textgreater{} 0 & Ereignisrisiko ist im Vergleich zur Referenzgruppe erhöht ✍ schlechteres Überleben (Risikoerhöhung) \\
HR \textless{} 0 & Ereignisrisiko ist im Vergleich zur Referenzgruppe reduziert ✍ besseres Überleben (Risikoreduktion) \\
\bottomrule()
\end{longtable}

\end{tablenos:no-prefix-table-caption}

Cox-Modelle wurden sowohl für die gesamte Patientenkohorte als auch für die Subgruppe mit HPV16 DNA-negativen Tumoren durchgeführt. Die Ergebnisse sind in zusammengefasst. Da bei den Patienten mit HPV16 DNA-negativen Tumoren kein lokales oder regionales Rezidiv eingetreten war, konnten für den primären Endpunkt in dieser Population keine Ergebnisse für die Cox-Regression ermittelt werden.

\textbf{Tabelle 5‑11: Ergebnisse der multivariaten Analyse von CSC-Marker und zusätzliche klinisch-pathologische Prognosefaktoren aller Patienten und für die Subgruppe mit HPV16 DNA-negativen Tumoren.} HR = Hazard Ratio; 95\% KI = 95\% Konfidenzintervall.

\begin{tablenos:no-prefix-table-caption}

\begin{longtable}[]{@{}
  >{\raggedright\arraybackslash}p{(\columnwidth - 12\tabcolsep) * \real{0.1429}}
  >{\raggedright\arraybackslash}p{(\columnwidth - 12\tabcolsep) * \real{0.1429}}
  >{\raggedright\arraybackslash}p{(\columnwidth - 12\tabcolsep) * \real{0.1429}}
  >{\raggedright\arraybackslash}p{(\columnwidth - 12\tabcolsep) * \real{0.1429}}
  >{\raggedright\arraybackslash}p{(\columnwidth - 12\tabcolsep) * \real{0.1429}}
  >{\raggedright\arraybackslash}p{(\columnwidth - 12\tabcolsep) * \real{0.1429}}
  >{\raggedright\arraybackslash}p{(\columnwidth - 12\tabcolsep) * \real{0.1429}}@{}}
\toprule()
\begin{minipage}[b]{\linewidth}\raggedright
Parameter
\end{minipage} & \begin{minipage}[b]{\linewidth}\raggedright
Loko-regionäre Kontrolle
\end{minipage} & \begin{minipage}[b]{\linewidth}\raggedright
\end{minipage} & \begin{minipage}[b]{\linewidth}\raggedright
Fernmetastasen-freies Überleben
\end{minipage} & \begin{minipage}[b]{\linewidth}\raggedright
\end{minipage} & \begin{minipage}[b]{\linewidth}\raggedright
Gesamtüberleben
\end{minipage} & \begin{minipage}[b]{\linewidth}\raggedright
\end{minipage} \\
\midrule()
\endhead
& HR (95\% KI) & p-Wert & HR (95\% KI) & p-Wert & HR (95\% KI) & p-Wert \\
Alle Patienten & & & & & & \\
CD44+* & 4,37 (0,57-33,5) & 0,155 & 1.68 (0,61-4,63) & 0,318 & 1,07 (0,52-2,20) & 0,866 \\
HPV16 DNA+** & 0,20 (0,05-0,91) & 0,038 & 0.49 (0,18-1,32) & 0,158 & 0,32 (0,15-0,67) & 0,003 \\
ECE-Status* & 1,56 (0,72-3,38) & 0,256 & 3.16 (1,50-6,64) & 0,002 & 1,71 (1,03-2,85) & 0,039 \\
Mundhöhlen CA & 2,11 (0,93-4,75) & 0,073 & 2.12 (0,96-4,70) & 0,064 & 1,68 (0,96-2,94) & 0,072 \\
Oropharynx CA & 0,73 (0,20-2,66) & 0,628 & 3.47 (1,44-8,36) & 0,005 & 0,74 (0,33-1,63) & 0,449 \\
Subgruppe mit HPV16 DNA-negativen Tumoren & & & & & & \\
CD44 & -- & --* & 1,64 (0,60-4,52) & 0,339 & 1,06 (0,51-2,19) & 0,872 \\
HPV16 DNA & & & & & & \\
ECE-Status & -- & -- & 3,12 (1,49-6,55) & 0,003 & 1,72 (1,03-2,88) & 0,037 \\
Mundhöhlen CA & -- & -- & 0,62 (0,27-1,42) & 0,257 & 2,27 (1,01-5,12) & 0,048 \\
Oropharynx CA & -- & -- & 0,30 (0,12-0,71) & 0,006 & 1,37 (0,62-3,05) & 0,435 \\
* Da es kein Ereignis in der CD44 negativen Gruppe gab (siehe Tabelle 5‑7) & & & & & & \\
** vs.~Other HPV- bzw. *CD44 - & & & & & & \\
\bottomrule()
\end{longtable}

\end{tablenos:no-prefix-table-caption}

In beiden Kohorten zeigten multivariaten Analysen für \textbf{CD44-Protein} keinen signifikanten Zusammenhang mit einem der drei Endpunkte. Patienten mit einem \textbf{HPV16} DNA-positiven Tumor hatten schätzungsweise ein um 80\% reduziertes Risiko für ein lokales oder regionales Rezidiv (HR 0,2; p=0,038) sowie ein um 68\% bessere Überlebenschance (HR 0,32; p=0,003) gegenüber Patienten mit einem HPV16 DNA-negativen Tumor (Referenzgruppe). Die multivariate Bewertung der sekundären Endpunkte ergab einen signifikanten Zusammenhang zwischen positiven \textbf{ECE-Status} mit dem Gesamtüberleben (OS) und mit Fernmetastasen-freies Überleben (DMS) sowohl in der Gesamtpopulation (DMS: HR 3,16, p=0,002; OS: HR 1,71, p=0,039) als auch in der Gruppe der HPV16 DNA-negativen Tumorkohorte (DMS: HR 3,12, p=0,003; OS: HR 1,72, p=0,037). \textbf{Oropharyngealkarzinome} hatten ein signifikant 3,47-fach erhöhtes Risiko für entfernte Metastasen in der Gesamtpopulation (HR 3,47, p=0,005). Wohingegen in der HPV16 DNA-negativen Kohorte das Risiko für Fernmetastasen bei Oropharyngealtumoren um 70\% reduziert war (HR 0,30, p=0,006). Die Tumorlokalisation in der \textbf{Mundhöhle} war ein grenzwertig signifikanter Einflussfaktor in Bezug auf das Gesamtüberleben in der HPV16 DNA-negativen Kohorte (HR 2,27, p=0,048), nicht aber in der gesamten Patientenpopulation.

\hypertarget{kapitel-5-diskussion}{%
\section{Kapitel 5: Diskussion}\label{kapitel-5-diskussion}}

\hypertarget{diskussion-der-methoden}{%
\subsection{Diskussion der Methoden}\label{diskussion-der-methoden}}

\hypertarget{diskussion-des-studiendesigns}{%
\subsubsection{Diskussion des Studiendesigns}\label{diskussion-des-studiendesigns}}

Für die Erfolgreiche Durchführung von Biomarker-Studien werden ausreichend große Patientenkohorten benötigt, die mit aktuellen Therapiestrategien beahndelt werden. Durch eine multizentrische nationale oder internationale Ausrichtung von Studien können eine höhere Rekrutierungszahl erreicht und die Zeiträume vo Studien verkürzt werden, sodass im Falle der retrospektiv durchgeführten Studien die Therapie der aktuellen Standardtherapie entspricht. Die multizentrisch generierten Studienergebnisse sind dabei besonders robust, weil Einzelzentrumseffekte und anderer Sektionsbias damit reduziert werden.

\hypertarget{tissue-microarrays}{%
\subsubsection{Tissue Microarrays}\label{tissue-microarrays}}

Die TMA ist eine geeignete Methode, um ein großes Gewebekollektiv zeitgleich und unter gleichen Bedingungen untersuchen zu können (\href{about:blank\#_ENREF_36}{Giltnane \& Rimm, 2004}; \href{about:blank\#_ENREF_46}{Kononen et al., 1998}). Es handelt sich um ein kostengünstiges und gewebesparendes Verfahren, das verschiedene biomolekulare Marker am gleichen Gewebekollektiv auf ihre Validität prüft. Die Identifizierung der einzelnen Gewebeproben in einem TMA-Block erfordert höchste Präzision und wird mittels einer Konstruktionsplanung erreicht, die die Lokalisation der einzelnen Fälle im TMA-Block erfasst. Auch die Auswertung der TMA-Präparate muss mit höchster Genauigkeit erfolgen, da jede Stanze die gleiche Form besitzt und es zu Fehlauswertungen kommen kann, z. B. durch Verrutschen des Präparats beim Mikroskopieren. Ein Defizit der TMA-Technik ist, dass nur ein geringer Teil des entnommenen Gewebes untersucht und erfasst wird. Dies könnte z. B. bei heterogenen Tumoren das Ergebnis verfälschen (\href{about:blank\#_ENREF_56}{Nocito et al., 2001}). Diese Fehlerquelle kann jedoch durch die Erhöhung der Stanzzahl pro Fall weitgehend minimiert werden. Da die Gewebeproben hinsichtlich Größe und Form vergleichbar sind, wird im Gegensatz zu konventionellen Schnittpräparaten eine subjektive Fehlbewertung durch die Betrachtung von unterschiedlich großen Gewebeproben vermieden.

\hypertarget{immunhistochemie}{%
\subsubsection{Immunhistochemie}\label{immunhistochemie}}

Die immunhistochemische Färbemethode ist standardisiertes Verfahren, welches in der histopathologischen Routinediagnostik verwendet wird.(\href{about:blank\#_ENREF_29}{DeLellis et al., 1979}). Die Fehlerquote ist gering, da die Färbeautomaten optimal gewartet sind und positive und negative Kontrollpräparate pro Färbereihe mitgeführt werden.90,89 Vielmehr sind die Ergebnisse vom Zustand des Gewebes, dem verwendeten Antikörper und den IHC-Bewertungskriterien abhängig.91 Nachteilig wirkt sich aus, dass das Gewebe durch die Parafineinbettung, die Formalinfixierung und auch durch eine ungeeignete Lagerung der Gewebeblöcke möglicherweise unzureichend erhalten bleibt und somit nicht untersucht werden kann bzw. zu falsch negativen Ergebnissen führt.70,91 Des Weiteren kann die Sensitivität der Antikörperklone von verschiedenen Firmen zu unterschiedlichen Ergebnissen führen und somit den retrospektiven Vergleich von Studien erschweren.89 Ein weiteres Problem liegt in der Auswertung und dem Vergleich von Färbeergebnissen anderer Studien, da jeder Pathologe subjektiv beurteilt und zum Teil andere Bewertungskriterien verwendet.89,66

Die Detektion der HPV-DNA mittels Polymerasekettenreaktion (PCR) ist eine häufig angewandte Methode, um eine HPV-Infektion zu diagnostizieren, sie ist jedoch nicht standardisiert.22,31 Die verschiedenen PCR- und Hybridisierungsverfahren weisen eine unterschiedliche Sensitivität und Spezifität auf, dies erschwert einen Vergleich der Ergebnisse von einzelnen Studien mit unterschiedlichen Verfahren.22 Durch die hohe Sensitivität der PCR muss bei der Durchführung der HPV-Analyse darauf geachtet werden, dass die Reaktionsgefäße und Pipetten steril sind, um einer möglichen Kontamination und damit falsch positiven Ergebnissen vorzubeugen.92 Als Vorteil dieser Methode gilt, dass sich damit unterschiedliche HPV-Subtypen ermitteln lassen.31 Nachteilig ist, dass eine bestimmte HPV-DNA- Basenlänge in einer Gewebeprobe vorhanden sein muss, um diese amplifizieren zu können.93 Jedoch ist die Qualität der Virus-DNA in den Gewebeproben durch die Vorbehandlung und die Langzeitarchivierung häufig unzureichend und eine Amplifizierung nur bedingt möglich.4,94

\hypertarget{diskussion-der-ergebnisse}{%
\subsection{Diskussion der Ergebnisse}\label{diskussion-der-ergebnisse}}

\hypertarget{einfluss-von-cd44-auf-das-klinische-outcome-von-lokal}{%
\subsubsection{Einfluss von CD44 auf das klinische Outcome von lokal}\label{einfluss-von-cd44-auf-das-klinische-outcome-von-lokal}}

Wir konnten zeigen, dass eine Überexpression von CD44 in lokal fortgeschrittenen Kopf- Hals-Plattenepithelkarzinomen mit einer schlechten loko-regionären Kontrolle nach postoperativer Radiochemotherapie assoziiert ist. Zudem stellt CD44 neben dem HPV- Status der Tumoren möglicherweise einen weiteren Stratifizierungsparameter zur Individualisierung der postoperativen Radiochemotherapie in Patienten mit Kopf-Hals- Plattenepithelkarzinomen dar.

Zur weiteren Evaluierung der klinischen Anwendbarkeit sollten die gewonnen Ergebnisse in prospektiven Validierungsstudien unter standardisierten Bedingungen weiter geprüft werden.

Zuvor haben wir und andere gezeigt, dass der HPV-Infektionsstatus ein starker Prognostiker für die lokoregionale Kontrolle bei Patienten mit lokal fortgeschrittenem HNSCC ist, die eine postoperative Radio(Chemotherapie (6, 28,~\textbf{\href{https://clincancerres.aacrjournals.org/content/22/11/2639\#ref-29}{29}}) mit einer verbesserten lokoregionalen Kontrolle und Radioempfindlichkeit von HPV16 DNA-positiv im Vergleich zu HPV16 DNA-negativen Tumoren~erhielten. In unserer vorherigen multizentrischen Bewertung fanden wir praktisch keine Rezidiven in der HPV16-DNA-positiven Gruppe bei Hochrisiko-HNSCC-Patienten, die PORT-C (6) erhielten. Patienten in dieser sehr vorteilhaften Gruppe können daher Kandidaten für potenzielle Strahlentherapie-Deeskalationsstudien sein. Für die weitere Schichtung von Patienten mit HPV-negativem HNSCC sind jedoch zusätzliche Biomarker erforderlich, um eine lokoregionale Kontrolle vorherzusagen, um zusätzliche Patienten zu identifizieren, die der sehr guten Prognosegruppe zugeordnet werden könnten, und um eine Patientengruppe mit sehr ungünstigem Ergebnis zu definieren, die ein Kandidat für Behandlungsverstärkungsstrategien sein

CD44 ist ein weithin erforschter CSC-Marker in HNSCC (\textbf{\href{https://clincancerres.aacrjournals.org/content/22/11/2639\#ref-30}{30}},~\textbf{\href{https://clincancerres.aacrjournals.org/content/22/11/2639\#ref-36}{36}},~\textbf{\href{https://clincancerres.aacrjournals.org/content/22/11/2639\#ref-37}{37}}). Eine Studie von de Jong und Kollegen zeigte, dass~\emph{CD44-}~und CD44-Proteinspiegel ein lokales Wiederauftreten nach Strahlentherapie bei Patienten mit Kehlkopfkrebs im Frühstadium signifikant vorhersagen (\textbf{\href{https://clincancerres.aacrjournals.org/content/22/11/2639\#ref-38}{38}}). Hier haben wir gezeigt, dass Patienten mit chirurgisch resezierten Tumoren ohne nachweisbare CD44-Proteinexpression die lokoregionalen Tumorkontrollraten im Vergleich zu Patienten mit CD44-Proteinexprimierenden Tumoren erhöht haben.

Vor kurzem wurde CD98 experimentell als mutmaßlicher CSC-Marker (\textbf{\href{https://clincancerres.aacrjournals.org/content/22/11/2639\#ref-39}{39}})~etabliert. In einer Studie mit 711 Patienten mit oropharyngealem Plattenepithelkarzinom wurde gezeigt, dass HPV-positive Tumore weniger CSC-Marker wie CD44 und CD98 exprimieren, während CD44- und CD98-Positivität mit signifikant geringerem progressionsfreien und OS (\textbf{\href{https://clincancerres.aacrjournals.org/content/22/11/2639\#ref-40}{40}})~verbunden war. In unserer Patientenkohorte fanden wir heraus, dass Patienten mit hoher Expression von~\emph{SLC3A2}, das für eines der CD98-Hetodimere kodiert, auch nach PORT-C eine schlechte lokoregionale Tumorkontrolle und ein erhöhtes Risiko für entfernte Metastasen aufweisen.

\hypertarget{hpv-status-als-weiteren-stratifizierungsparameter}{%
\subsubsection{HPV-Status als weiteren Stratifizierungsparameter}\label{hpv-status-als-weiteren-stratifizierungsparameter}}

Während mehrere frühere Studien starke Beweise dafür lieferten, dass der HPV-Status ein signifikanter prognostischer Marker für die lokoregionale Tumorkontrolle und/oder das Überleben bei Patienten ist, die mit primärer Strahlentherapie oder Radiochemotherapie für lokal fortgeschrittenes HNSCC {[}14{]}, {[}16{]}, {[}18{]}, {[}19{]} behandelt werden, sind die Auswirkungen des HPV-Status auf das Ergebnis einer postoperativen Radio(Chemotherapie weniger gut untersucht. Die Ergebnisse der vorliegenden multizentrischen retrospektiven Studie des DKTK-ROG zeigen, dass die HPV16-DNA-Positivität ein signifikanter Prognostiker der lokoregionalen Tumorkontrolle und des Überlebens von Patienten ist, die nach einer chirurgischen Resektion von lokal fortgeschrittenem HNSCC mit einer postoperativer Radiochemotherapie auf Die Wirkung scheint in allen Behandlungszentren robust zu sein und wird durch die Ergebnisse von Oropharyngealkrebs angetrieben. Unsere Ergebnisse stehen im Einklang mit einer Studie von Snietura et al., die den Einfluss einer HPV-Infektion auf das klinische Ergebnis in einer Post-hoc-Analyse einer randomisierten klinischen Studie mit zwei verschiedenen PORT-Schemas ohne Chemotherapie bei 279 HNSCC-Patienten untersuchte. Die HPV-Analyse wurde bei Tumoren von 131 Patienten durchgeführt. Von den 66 Patienten mit oraler Kavität oder oropharyngealem Karzinom erwiesen sich 9 als positiv für HPV16-DNA und wurden nach 5 Jahren lokal kontrolliert, während die lokoregionale Tumorkontrollrate in der gesamten HPV-DNA-Negativgruppe nur 58 \% betrug {[}24{]}. Zusammengenommen scheint die HPV16-DNA ein potenziell vielversprechender Biomarker für die Schichtung und individuelle Verschreibung der postoperativen Strahlentherapie zu sein. Die HPV-Positivität scheint ausreichen, um eine Patientenkohorte zu definieren, die nach PORT-C höchst unwahrscheinlich lokoregionale Rezidivenzen entwickelt, was im Gegensatz zur primären Strahlenchemotherapie steht, bei der mehr Schichtungsparameter erforderlich sind {[}25{]}. Dieser Unterschied zwischen den beiden Behandlungsansätzen kann dadurch verursacht werden, dass solche zusätzlichen Faktoren, am offensichtlichsten Tumorvolumen, bei der Reseziert des Tumors eine geringere Rolle spielen. Andere patientenbezogene Risikofaktoren wie der Raucherstatus konnten in unserem Datensatz nicht ausgewertet werden, könnten aber auch relevant sein. Wir führen derzeit ähnliche Analysen in einer Patientenkohorte durch, die von denselben Zentren und innerhalb desselben Zeitraums mit der primären Radiochemotherapie behandelt wurde, um solche Differentialprognosen mit einem mehrdimensionalen statistischen Ansatz einschließlich radiobiologischer Schätzungen, z.B. zur Tumorzellzahl, weiter zu bewerten. Für die Gruppe der HPV-negativen Patienten ist die Situation weitgehend anders. Hier kann HPV nicht als einziger Biomarker verwendet werden, um Tumorrezidiven vorherzusagen, wie die geringe Empfindlichkeit von 38 \% oder 41 \% für HPV16-DNA oder p16-Positivität zeigt. Daher benötigt die HPV-negative Gruppe weitere Untersuchungen potenzieller Biomarker, die für Patienten, die möglicherweise eine Intensivierung der Behandlung benötigen, und für Patienten, für die kein lokales Wiederauftreten zu erwarten ist, schichten.

Derzeit läuft eine prospektive multizentrische Studie des DKTK-ROG, um den prognostischen Wert der HPV16-DNA-Positivität für die lokoregionale Tumorkontrolle nach PORT-C bei 240 HNSCC-Patienten zu validieren. Wenn die Ergebnisse dieser retrospektiven Kohorte bestätigt werden, wird eine interventionelle Studie zur Deeskalation der PORT-C-Strahlungsdosen bei HPV16-DNA-positiven, klinisch geeigneten Oropharyngealkrebspatienten eingeleitet. Die Spezifitätsbewertung der HPV16-DNA-Positivität für die lokoregionale Tumorkontrolle aus dieser Untersuchung legt nahe, dass bei einer moderaten Abnahme der Strahlendosis bei diesen Patienten nur sehr wenige, wenn überhaupt, Wiederauftreten zu erwarten sind, daher können in einer solchen Studie strenge Stoppregeln für die Patientensicherheit gegen das Risiko einer minderwertigen Behandlung angewendet werden, z.B. mit einem Pocock-Grenzansatz {[}26{]}. Eine weitere Verfeinerung der Risikoschichtung speziell für die HPV-negative Gruppe kann sich aus einer prospektiven Bewertung klinischer Parameter {[}27{]} in der Validierungsstudie und aus den laufenden Bemühungen ergeben, weitere Biomarker in der aktuellen Retrospektive und in der Validierungspatientenkohorte zu identifizieren.

Das geringe Risiko eines lokoregionalen Wiederauftretens bei HPV16-DNA-positiven Oropharyngealkarzinomen nach kurativ beabsichtigter Resektion und PORT-C deutet darauf hin, dass entweder zu Beginn von PORT-C weniger Tumorstammzellen vorhanden sind, dass die verbleibenden HPV-positiven Tumorzellen radio-(chemo-)empfindlicher sind oder eine Kombination aus beidem. Kürzlich wurde berichtet, dass HPV-positive Oropharyngealkrebs eine geringe Expression von Stammzellmarkern wie CD44 und CD98 im Vergleich zu HPV-negativen Oropharyngealkarzinomen zeigt {[}28{]}. Darüber hinaus zeigten Patienten mit HPV-positiven und niedrigen CD98-exprimierenden Tumoren ein besseres Gesamtüberleben und ein progressionsfreies Überleben im Vergleich zu Patienten mit hohen CD98-exprimierenden HPV-positiven Tumoren. Die erhöhte Radioempfindlichkeit von HPV-positiven Tumorzellen wird durch eine Reihe von Untersuchungen unterstützt. HPV-positive HNSCC-Zelllinien (alle positiv für HPV-DNA, HPV-RNA und p16), die durch einen Koloniebildungstest in vitro bewertet wurden, zeigten aufgrund der beeinträchtigten DNA-Reparaturkapazität eine höhere zelluläre Radioempfindlichkeit im Vergleich zu HPV-negativen Zelllinien {[}29{]}. Ähnliche Beobachtungen wurden von anderen berichtet {[}30{]},{[}31{]}. Weitere Beobachtungen, die sowohl In-vitro- als auch in In-vivo-Ansätze verwenden, deuten darauf hin, dass überexprimiertes p16 die Rekrutierung von RAD51 an die DNA-Schädensstelle in HPV-positivem HNSCC durch Downregulation von Cyclin D1 beeinträchtigt und damit den Zellzyklus und die homologe rekombinations

Es gibt derzeit keinen allgemein vereinbarten Konsens für die Bewertung des HPV-Infektionsstatus als potenzieller Biomarker; allgemeine Methoden zur Bewertung der HPV-Infektion umfassen HPV-DNA, HPV-RNA und p16-Überexpression {[}33{]}, {[}34{]}, {[}35{]}. Die überwiegende Mehrheit des HPV-positiven HNSCC hat sich als positiv für HPV16-DNA {[}21{]}, {[}36{]} erwiesen, was den hier berichteten Ergebnissen entspricht. HPV16-DNA zeigte stärkere Korrelationen mit den Ergebnisparametern im Vergleich zur p16-Immunhistochemie in einer Kohorte von 50 Patienten mit oropharyngealen Tumoren, die eine primäre Radiochemotherapie erhielten {[}37{]}. Auch in unserer Studie erscheint HPV16 DNA als stärkerer Prognostiker für die lokoregionale Tumorkontrolle im Vergleich zur p16-Expression (Tabelle 4 vs.~Ergänzende Tabelle S2), dies muss jedoch in einer größeren Kohorte validiert werden.

In der vorliegenden Studie war HPV16 DNA-Positivität ein starker unabhängiger Prognostiker für die lokoregionale Tumorkontrolle bei Oropharyngeal-, aber nicht bei Mundhöhlentumoren. Im Gegensatz dazu wurde bei Mundhöhlentumoren eine erhöhte p53-Positivität beobachtet, was auf einen alternativen Weg für die Tumorentwicklung hindeutet, z.B. Lebensstilfaktoren. Es ist bekannt, dass das Tumorsuppressorgen TP53 an der Karzinogenese von HNSCC beteiligt ist {[}38{]}, und seine Überexpression wird bei starken Rauchern und schweren Trinkern berichtet {[}39{]}, {[}40{]}. Erhöhte Positivität wurde mit TP53-Genmutationen in Verbindung gebracht, die zur Stabilisierung und nuklearen Akkumulation von p53-Proteinen führen können{[}41{]}. Es wurde gezeigt, dass das HPV-Onkoprotein E6 p53 {[}42{]}, {[}43{]} inaktiviert und hemmt, was der Tatsache entspricht, dass der Großteil unserer HPV-positiven Studienkohorte für p53 negativ war.

In unserer Studie ist die extrakapseuläre Verlängerung der Lymphknoten ein prognostischer Faktor für das Gesamtüberleben bei Patienten mit oralen Hohlraumkarzinomen, aber nicht in der gesamten Patientenpopulation. Dies scheint im Gegensatz zu den Ergebnissen der Metaanalyse von Bernier et al.~zu stehen, die zeigt, dass positive Margen und/oder extrakapseuläre Verlängerung die wichtigsten Prognosen für ein schlechtes Ergebnis/Gesamtüberleben sind {[}9{]}. Diese Metaanalyse schichtet jedoch nicht zwischen Mundhöhle und Oropharyngealkrebs, und molekulare Biomarker wurden in den Studien, die in der Metaanalyse enthalten waren, nicht als verwirrende Faktoren untersucht (EORTC- und RTOG-Studien). Darüber hinaus könnten Patienten, die in die EORTC- und RTOG-Studien einbezogen werden, eine andere Population widerspiegeln als die in dieser Studie analysierte Patientenkohorte, was die Notwendigkeit einer konstanten Markeranpassung für die Patientenschichtung unterstreicht. Weitere Bemühungen zur Untersuchung verschiedener Biomarker speziell für die HPV-negative Gruppe, die eine postoperative Radiochemotherapie erhält, sowie für die primäre Radiochemotherapie sind derzeit im DKTK-ROG unter Verwendung von Material der hier berichteten Kohorte und der Validierungsstudie im Gange.

Abschließend zeigen unsere Ergebnisse dieser retrospektiven explorativen multizentrischen Studie, dass die HPV16-DNA nach postoperativer Cisplatin-basierter Radiochemotherapie des lokal fortgeschrittenen Oropharyngealkarzinoms ein starker Prognostiker der lokoregionalen Tumorkontrolle zu sein scheint und daher ein vielversprechender Biomarker für die Patientenschicht Die Wirkung schien über die 8 Behandlungszentren robust zu sein. Bei Patienten mit HPV16 DNA-positivem oropharyngealem Karzinom kann die Deintensifikation eine gültige interventionelle Option für eine derzeit vorbereitete prospektive Studie sein.

Zuvor haben wir und andere gezeigt, dass der HPV-Infektionsstatus ein starker Prognostiker für die lokoregionale Kontrolle bei Patienten mit lokal fortgeschrittenem HNSCC ist, die eine postoperative Radio(Chemotherapie (6, 28, 29) mit einer verbesserten lokoregionalen Kontrolle und Radioempfindlichkeit von HPV16 DNA-positiv im Vergleich zu HPV In unserer vorherigen multizentrischen Bewertung fanden wir praktisch keine Rezidiven in der HPV16-DNA-positiven Gruppe bei Hochrisiko-HNSCC-Patienten, die PORT-C (6) erhielten. Patienten in dieser sehr vorteilhaften Gruppe können daher Kandidaten für potenzielle Strahlentherapie-Deeskalationsstudien sein. Für die weitere Schichtung von Patienten mit HPV-negativem HNSCC sind jedoch zusätzliche Biomarker erforderlich, um eine lokoregionale Kontrolle vorherzusagen, um zusätzliche Patienten zu identifizieren, die der sehr guten Prognosegruppe zugeordnet werden könnten, und um eine Patientengruppe mit sehr ungünstigem Ergebnis zu definieren, die ein Kandidat für Behandlungsverstärkungsstrategien sein könnte.

Der Sauerstoffversorgungsstatus der Tumore ist einer der bekannten Biomarker für das Ergebnis der primären Strahlentherapie in HNSCC, d.h. von makroskopischen Tumoren, die keiner chirurgischen Entfernung unterzogen wurden. Hier wollten wir die Relevanz des Hypoxiestatus anhand der Hypoxie-Gensignaturen von Toustrup und Kollegen (15) und Eustace und Kollegen (16) in der Tumorprobe bestimmen, die durch chirurgische Resektion vor PORT-C erhalten wurde. Wir fanden heraus, dass der Hypoxiestatus vor der Behandlung bei Patienten mit PORT-C prognostisch war. Patienten mit hohen Hypoxiespiegeln in der vor PORT-C erhaltenen chirurgischen Probe hatten im Vergleich zu Patienten mit niedrigen hypoxischen Tumoren eine schlechte lokoregionale Tumorkontrolle. Dieser Effekt führt hauptsächlich zu der Untergruppe von HPV-negativem HNSCC. Unter Berücksichtigung der Tatsache, dass Hypoxie in der chirurgischen Probe untersucht und die Radiochemotherapie nach der Operation auf die potenziellen Resttumorzellen angewendet wurde, ist die Feststellung eines signifikanten Zusammenhangs von Hypoxie innerhalb des resezierten Primärtumors mit dem Ergebnis von PORT-C auf den ersten Blick überraschend. Da es unwahrscheinlich ist, dass sich die geringe Anzahl von Resttumorzellen nach der Operation auch in der Hypoxie unterscheidet, deutet diese Erkenntnis darauf hin, dass Hypoxie das Ergebnis der Strahlentherapie nicht nur durch eine direkte biochemische oder radiobiologische Wirkung auf die zelluläre Radioresistenz des Tumors, sondern auch durch andere radiobiologische Mechanismen beeinflusst. Dies würde mit früheren experimentellen Daten aus unserem Labor übereinstimmen, die zeigen, dass die Hypoxie vor der Behandlung die lokale Tumorkontrolle nach der Strahlentherapie auch dann beeinflusst, wenn die Strahlentherapie unter homogenen anoxischen Bedingungen angewendet wurde (20). Da jüngste Studien darauf hindeuteten, dass Tumorhypoxie auch Stamm- und invasives Wachstum als externen Faktor begünstigt (überprüft in Ref. 30--33), untersuchten wir die Rolle potenzieller CSC-Marker in unserer Kohorte. Es ist bekannt, dass CSCs eine wichtige Rolle bei der Radioresistenz spielen (überprüft in Ref. 34), und eine hochhypoxische Tumormikroumgebung trägt zu einem erhöhten clonogenen Potenzial bei (20, 35).

CD44 ist ein weithin erforschter CSC-Marker in HNSCC (30, 36, 37). Eine Studie von de Jong und Kollegen zeigte, dass CD44- und CD44-Proteinspiegel ein lokales Wiederauftreten nach Strahlentherapie bei Patienten mit Kehlkopfkrebs im Frühstadium signifikant vorhersagen (38). Hier haben wir gezeigt, dass Patienten mit chirurgisch resezierten Tumoren ohne nachweisbare CD44-Proteinexpression die lokoregionalen Tumorkontrollraten im Vergleich zu Patienten mit CD44-Proteinexprimierenden Tumoren erhöht haben.

Vor kurzem wurde CD98 experimentell als mutmaßlicher CSC-Marker (39) etabliert. In einer Studie mit 711 Patienten mit oropharyngealem Plattenepithelkarzinom wurde gezeigt, dass HPV-positive Tumore weniger CSC-Marker wie CD44 und CD98 exprimieren, während CD44- und CD98-Positivität mit signifikant geringerem progressionsfreien und OS (40) verbunden war. In unserer Patientenkohorte fanden wir heraus, dass Patienten mit hoher Expression von SLC3A2, das für eines der CD98-Hetodimere kodiert, auch nach PORT-C eine schlechte lokoregionale Tumorkontrolle und ein erhöhtes Risiko für entfernte Metastasen aufweisen.

Es wurde gezeigt, dass der MET-Weg die Selbsterneuerung und Tumorgenität in stammähnlichen HNSCC-ähnlichen Zellen fördert (41), und es wurde gezeigt, dass die pharmakologische selektive Hemmung der MET zur Elimination von CSCs führt (42). Darüber hinaus wurde festgestellt, dass MET mit einer schlechten Prognose bei Patienten mit lokal fortgeschrittenem p16-negativem HNSCC verbunden ist, die mit einer primären Radiochemotherapie behandelt wurden (43). In der aktuellen Studie haben wir gezeigt, dass MET-überexprimierende HPV16-DNA-negative HNSCC mit einer schlechten lokoregionalen Tumorkontrolle und erhöhten entfernten Metastasen nach PORT-C verbunden sind. Interessanterweise zeigten Pennacchietti und Kollegen, dass Hypoxie die Transkription des MET-Protoonkogens aktiviert, dass die MET-Überexpression mit hypoxischen Tumorbereichen verbunden ist und dass die MET-Hemmung ein hypoxiebedingtes Zellwachstum verhindert (44). Unsere Daten, die eine positive Korrelation der MET mit Tumorhypoxie zeigen, unterstützen diesen Zusammenhang beider Faktoren. Dennoch zeigte die relativ geringe Anzahl niedriger hypoxischer Tumoren in der HPV16 DNA-negativ/MET-positiven Gruppe eine signifikant höhere lokoregionale Kontrollrate im Vergleich zu den hochhypoxischen MET-positiven Tumoren. Darüber hinaus deuten unsere Daten darauf hin, dass SLC3A2 und MET bei Patienten mit HPV16 DNA-negativen Tumoren geeignete Marker sein können, um eine Gruppe von Patienten zu bestimmen, die eine sehr schlechte Prognose haben und daher von der Intensivierung der Behandlung profitieren können.

Obwohl die in dieser Studie verwendeten Hypoxieprofile für die lokoregionale Tumorkontrolle prognostisch waren, waren sie kein Hinweis auf ein erhöhtes Risiko entfernter Metastasen (Tabelle 1). Dies erscheint angesichts experimenteller und klinischer Daten, die zeigen, dass Hypoxie entfernte Metastasen antreiben kann (31, 45--47), kontraintuitiv. Nach unserem besten Wissen wurde dies jedoch bisher im HNSCC im postoperativen Umfeld nicht untersucht. Einige der Gene in den Hypoxieprofilen, die in unserer Studie verwendet werden, könnten auch mit anderen biologischen Phänomenen in Verbindung gebracht werden, einschließlich der Stammheit von Krebszellen. Die Korrelationskoeffizienten zwischen den verschiedenen CSC-Markern und Hypoxieprofilen (Tabelle 3) deuten darauf hin, dass, obwohl eine signifikante Korrelation vorliegt, beide Parameter weitgehend unabhängig voneinander ausgedrückt werden. Im Gegensatz zu den Hypoxieprofilen korrelierte die CSC-Markerexpression in unserer Studie nicht nur mit der lokoregionalen Kontrolle, sondern auch mit entfernten Metastasen. Die Expression von CSC-Markern wurde als potenzielles Surrogat der CSC-Dichte vorgeschlagen, d.h. der Anzahl der Zellen, die pro gegebenem Tumorgewebevolumen metastasieren können (30, 34, 36, 46, 48, 49). Weitere mechanistische Untersuchungen des Zusammenhangs von Hypoxie, Stammheit und metastasierendem Risiko in HNSCC in Korrelation mit klinischen Daten scheinen daher ein interessanter Weg für eine bessere Schichtung der Patienten für systemische Therapien und für die Entdeckung neuer Ziele zu sein. Eine wichtige Besonderheit unserer Studie ist, dass Biomarker in einer Kohorte von Patienten untersucht wurden, die postoperativ per Strahlenchemotherapie behandelt wurden, mit insgesamt höheren lokoregionalen Kontrollraten im Vergleich zu Studien zur Bewertung der primären Strahlenchemotherapie bei HPV-negativem oder -positivem HNSCC (50). Derzeit ist nicht bekannt, ob sich das prädiktive Potenzial von Biomarkern für die verschiedenen klinischen Risikogruppen unterscheidet, die eine postoperative oder primäre Radiochemotherapie erhalten. Um diese Frage anzugehen, wird die letztere Risikogruppe von Patienten derzeit mit den gleichen Biomarkern bewertet, die hier vom DKTK-ROG untersucht wurden.

Zusammengenommen ist dies die erste systematische multizentrische Analyse, die einen Zusammenhang zwischen hohen Konzentrationen von Tumorhypoxie und CSC-Markerexpression innerhalb des chirurgisch entfernten Primärtumors mit beeinträchtigter lokoregionaler Tumorkontrolle nach PORT-C in HNSCC zeigt. In HPV16 DNA-negativem HNSCC kann die Negativität für diese Parameter helfen, eine Untergruppe von Patienten mit lokoregionalen Kontrollraten zu identifizieren, die so hoch sind wie bei HPV16 DNA-positiven Tumoren beobachtet. Darüber hinaus scheinen Hypoxie- und CSC-Marker-positive Tumoren eine Untergruppe von Patienten darzustellen, die mit dem aktuellen Standard-PORT-C unterbehandelt ist. Nach der Validierung in einer derzeit laufenden prospektiven Studie können diese Parameter dazu beitragen, Patienten für individualisierte Behandlungsdeeskalations- oder Intensivierungsstrategien weiter zu schichten.

\hypertarget{kapitle-8-literaturverzeichnis}{%
\section{Kapitle 8: Literaturverzeichnis}\label{kapitle-8-literaturverzeichnis}}

Uncategorized References

Ailles, L., \& Prince, M. (2009). Cancer stem cells in head and neck squamous cell carcinoma. \emph{Methods Mol Biol}, \emph{568}, 175-193. \url{https://doi.org/10.1007/978-1-59745-280-9_11}

Al-Hajj, M., Wicha, M. S., Benito-Hernandez, A., Morrison, S. J., \& Clarke, M. F. (2003). Prospective identification of tumorigenic breast cancer cells. \emph{Proc Natl Acad Sci U S A}, \emph{100}(7), 3983-3988. \url{https://doi.org/10.1073/pnas.0530291100}

Amin, M. B., Edge, S. B., Greene, F. L., Byrd, D. R., Brookland, R. K., Washington, M. K., Gershenwald, J. E., Compton, C. C., Hess, K. R., \& Sullivan, D. C. (2018). \emph{AJCC Cancer Staging Manual}. Springer International Publishing. \url{https://books.google.de/books?id=O2PyjwEACAAJ}

Andl, T., Kahn, T., Pfuhl, A., Nicola, T., Erber, R., Conradt, C., Klein, W., Helbig, M., Dietz, A., Weidauer, H., \& Bosch, F. X. (1998). Etiological involvement of oncogenic human papillomavirus in tonsillar squamous cell carcinomas lacking retinoblastoma cell cycle control. \emph{Cancer Res}, \emph{58}(1), 5-13. \url{https://www.ncbi.nlm.nih.gov/pubmed/9426048}

Ang, K. K., Harris, J., Wheeler, R., Weber, R., Rosenthal, D. I., Nguyen-Tan, P. F., Westra, W. H., Chung, C. H., Jordan, R. C., Lu, C., Kim, H., Axelrod, R., Silverman, C. C., Redmond, K. P., \& Gillison, M. L. (2010). Human papillomavirus and survival of patients with oropharyngeal cancer. \emph{N Engl J Med}, \emph{363}(1), 24-35. \url{https://doi.org/10.1056/NEJMoa0912217}

Ang, K. K., Harris, J., Wheeler, R., Weber, R., Rosenthal, D. I., Nguyen-Tân, P. F., Westra, W. H., Chung, C. H., Jordan, R. C., Lu, C., Kim, H., Axelrod, R., Silverman, C. C., Redmond, K. P., \& Gillison, M. L. (2010). Human papillomavirus and survival of patients with oropharyngeal cancer. \emph{N Engl J Med}, \emph{363}(1), 24-35. \url{https://doi.org/10.1056/NEJMoa0912217}

Argiris, A., Karamouzis, M. V., Raben, D., \& Ferris, R. L. (2008). Head and neck cancer. \emph{Lancet}, \emph{371}(9625), 1695-1709. \url{https://doi.org/10.1016/s0140-6736(08)60728-x}

Barnes, L., Pathologie, U.-S. Z. D., Eveson, J. W., Pathology, I. A. o., Sidransky, D., Organization, W. H., Cancer, I. A. f.~R. o., \& Reichart, P. (2005). \emph{Pathology and Genetics of Head and Neck Tumours}. IARC Press. \url{https://books.google.de/books?id=mrm8hxiJ4XIC}

Baumann, M., \& Krause, M. (2010). CD44: a cancer stem cell-related biomarker with predictive potential for radiotherapy. \emph{Clin Cancer Res}, \emph{16}(21), 5091-5093. \url{https://doi.org/10.1158/1078-0432.CCR-10-2244}

Bernier, J., Cooper, J. S., Pajak, T. F., van Glabbeke, M., Bourhis, J., Forastiere, A., Ozsahin, E. M., Jacobs, J. R., Jassem, J., Ang, K. K., \& Lefebvre, J. L. (2005). Defining risk levels in locally advanced head and neck cancers: a comparative analysis of concurrent postoperative radiation plus chemotherapy trials of the EORTC (\#22931) and RTOG (\# 9501). \emph{Head Neck}, \emph{27}(10), 843-850. \url{https://doi.org/10.1002/hed.20279}

Biomarkers Definitions Working, G. (2001). Biomarkers and surrogate endpoints: preferred definitions and conceptual framework. \emph{Clin Pharmacol Ther}, \emph{69}(3), 89-95. \url{https://doi.org/10.1067/mcp.2001.113989}

Böcker, W. (2008). \emph{Pathologie: mit über 200 Tabellen}. Elsevier, Urban \& Fischer. \url{https://books.google.de/books?id=_aKIKlvq3KoC}

Bonnet, D., \& Dick, J. E. (1997). Human acute myeloid leukemia is organized as a hierarchy that originates from a primitive hematopoietic cell. \emph{Nat Med}, \emph{3}(7), 730-737. \url{https://doi.org/10.1038/nm0797-730}

Bootz, F. (2020). {[}Guideline on diagnosis, treatment, and follow-up of laryngeal cancer{]}. \emph{Radiologe}, \emph{60}(11), 1052-1057. \url{https://doi.org/10.1007/s00117-020-00760-9} (S3-Leitlinie Diagnostik, Therapie und Nachsorge des Larynxkarzinoms.)

Bouvard, V., Baan, R., Straif, K., Grosse, Y., Secretan, B., El Ghissassi, F., Benbrahim-Tallaa, L., Guha, N., Freeman, C., Galichet, L., Cogliano, V., \& Group, W. H. O. I. A. f.~R. o. C. M. W. (2009). A review of human carcinogens--Part B: biological agents. \emph{Lancet Oncol}, \emph{10}(4), 321-322. \url{https://doi.org/10.1016/s1470-2045(09)70096-8}

Cabrera Rodriguez, J., Cacicedo, J., Giralt, J., Garcia Miragall, E., Lloret, M., Arias, F., Gonzalez Ruiz, M. A., \& Contreras, J. (2018). GEORCC recommendations on target volumes in radiotherapy for Head Neck Cancer of Unkown Primary. \emph{Crit Rev Oncol Hematol}, \emph{130}, 51-59. \url{https://doi.org/10.1016/j.critrevonc.2018.07.006}

Cardesa, A., Remmele, W., Klöppel, G., Mentzel, T., Kreipe, H. H., Rudolph, P., \& Slootweg, P. (2008). \emph{Pathologie: Kopf-Hals-Region, Weichgewebstumoren, Haut}. Springer Berlin Heidelberg. \url{https://books.google.de/books?id=IBJSDIxrSUUC}

Carvalho, A. L., Nishimoto, I. N., Califano, J. A., \& Kowalski, L. P. (2005). Trends in incidence and prognosis for head and neck cancer in the United States: a site-specific analysis of the SEER database. \emph{Int J Cancer}, \emph{114}(5), 806-816. \url{https://doi.org/10.1002/ijc.20740}

Cawson, R. A., \& Odell, E. W. (2008). \emph{Cawson's Essentials of Oral Pathology and Oral Medicine E-Book}. Elsevier Health Sciences. \url{https://books.google.de/books?id=K035mlSXAAsC}

Cawson, R. A., \& Odell, E. W. (2017). \emph{Cawson's Essentials of Oral Pathology and Oral Medicine}. Elsevier Health Sciences UK. \url{https://books.google.de/books?id=pAdLzQEACAAJ}

Chen, Y. W., Chen, K. H., Huang, P. I., Chen, Y. C., Chiou, G. Y., Lo, W. L., Tseng, L. M., Hsu, H. S., Chang, K. W., \& Chiou, S. H. (2010). Cucurbitacin I suppressed stem-like property and enhanced radiation-induced apoptosis in head and neck squamous carcinoma--derived CD44(+)ALDH1(+) cells. \emph{Mol Cancer Ther}, \emph{9}(11), 2879-2892. \url{https://doi.org/10.1158/1535-7163.MCT-10-0504}

Chin, D., Boyle, G. M., Porceddu, S., Theile, D. R., Parsons, P. G., \& Coman, W. B. (2006). Head and neck cancer: past, present and future. \emph{Expert Rev Anticancer Ther}, \emph{6}(7), 1111-1118. \url{https://doi.org/10.1586/14737140.6.7.1111}

Chung, C. H., \& Gillison, M. L. (2009). Human papillomavirus in head and neck cancer: its role in pathogenesis and clinical implications. \emph{Clin Cancer Res}, \emph{15}(22), 6758-6762. \url{https://doi.org/10.1158/1078-0432.CCR-09-0784}

Clarke, A. R., \& Meniel, V. (2006). The intestinal stem cell niche studied through conditional transgenesis. \emph{Ernst Schering Found Symp Proc}(5), 99-108. \url{https://doi.org/10.1007/2789_2007_046}

Cooper, J. S., Zhang, Q., Pajak, T. F., Forastiere, A. A., Jacobs, J., Saxman, S. B., Kish, J. A., Kim, H. E., Cmelak, A. J., Rotman, M., Lustig, R., Ensley, J. F., Thorstad, W., Schultz, C. J., Yom, S. S., \& Ang, K. K. (2012). Long-term follow-up of the RTOG 9501/intergroup phase III trial: postoperative concurrent radiation therapy and chemotherapy in high-risk squamous cell carcinoma of the head and neck. \emph{Int J Radiat Oncol Biol Phys}, \emph{84}(5), 1198-1205. \url{https://doi.org/10.1016/j.ijrobp.2012.05.008}

Curado, M. P., \& Boyle, P. (2013). Epidemiology of head and neck squamous cell carcinoma not related to tobacco or alcohol. \emph{Curr Opin Oncol}, \emph{25}(3), 229-234. \url{https://doi.org/10.1097/CCO.0b013e32835ff48c}

D'Souza, G., Agrawal, Y., Halpern, J., Bodison, S., \& Gillison, M. L. (2009). Oral sexual behaviors associated with prevalent oral human papillomavirus infection. \emph{J Infect Dis}, \emph{199}(9), 1263-1269. \url{https://doi.org/10.1086/597755}

de Jong, M. C., Pramana, J., van der Wal, J. E., Lacko, M., Peutz-Kootstra, C. J., de Jong, J. M., Takes, R. P., Kaanders, J. H., van der Laan, B. F., Wachters, J., Jansen, J. C., Rasch, C. R., van Velthuysen, M. L., Grenman, R., Hoebers, F. J., Schuuring, E., van den Brekel, M. W., \& Begg, A. C. (2010). CD44 expression predicts local recurrence after radiotherapy in larynx cancer. \emph{Clin Cancer Res}, \emph{16}(21), 5329-5338. \url{https://doi.org/10.1158/1078-0432.CCR-10-0799}

DeLellis, R. A., Sternberger, L. A., Mann, R. B., Banks, P. M., \& Nakane, P. K. (1979). Immunoperoxidase technics in diagnostic pathology. Report of a workshop sponsored by the National Cancer Institute. \emph{Am J Clin Pathol}, \emph{71}(5), 483-488. \url{https://doi.org/10.1093/ajcp/71.5.483}

Duvvuri, U., \& Myers, J. N. (2009). Contemporary management of oropharyngeal cancer: anatomy and physiology of the oropharynx. \emph{Curr Probl Surg}, \emph{46}(2), 119-184. \url{https://doi.org/10.1067/j.cpsurg.2008.10.003}

El-Naggar, A. K., Chan, J. K. C., Grandis, J. R., Takata, T., \& Slootweg, P. J. (2017). \emph{WHO Classification of Head and Neck Tumours}. International Agency for Research on Cancer. \url{https://books.google.de/books?id=EDo5MQAACAAJ}

Fakhry, C., Westra, W. H., Li, S., Cmelak, A., Ridge, J. A., Pinto, H., Forastiere, A., \& Gillison, M. L. (2008). Improved survival of patients with human papillomavirus-positive head and neck squamous cell carcinoma in a prospective clinical trial. \emph{J Natl Cancer Inst}, \emph{100}(4), 261-269. \url{https://doi.org/10.1093/jnci/djn011}

Ferlay, J., Colombet, M., Soerjomataram, I., Parkin, D. M., Pineros, M., Znaor, A., \& Bray, F. (2021). Cancer statistics for the year 2020: An overview. \emph{Int J Cancer}. \url{https://doi.org/10.1002/ijc.33588}

Gillison, M. L., D'Souza, G., Westra, W., Sugar, E., Xiao, W., Begum, S., \& Viscidi, R. (2008). Distinct risk factor profiles for human papillomavirus type 16-positive and human papillomavirus type 16-negative head and neck cancers. \emph{J Natl Cancer Inst}, \emph{100}(6), 407-420. \url{https://doi.org/10.1093/jnci/djn025}

Gillison, M. L., Koch, W. M., Capone, R. B., Spafford, M., Westra, W. H., Wu, L., Zahurak, M. L., Daniel, R. W., Viglione, M., Symer, D. E., Shah, K. V., \& Sidransky, D. (2000). Evidence for a causal association between human papillomavirus and a subset of head and neck cancers. \emph{J Natl Cancer Inst}, \emph{92}(9), 709-720. \url{https://doi.org/10.1093/jnci/92.9.709}

Giltnane, J. M., \& Rimm, D. L. (2004). Technology insight: Identification of biomarkers with tissue microarray technology. \emph{Nat Clin Pract Oncol}, \emph{1}(2), 104-111. \url{https://doi.org/10.1038/ncponc0046}

Ginestier, C., Hur, M. H., Charafe-Jauffret, E., Monville, F., Dutcher, J., Brown, M., Jacquemier, J., Viens, P., Kleer, C. G., Liu, S., Schott, A., Hayes, D., Birnbaum, D., Wicha, M. S., \& Dontu, G. (2007). ALDH1 is a marker of normal and malignant human mammary stem cells and a predictor of poor clinical outcome. \emph{Cell Stem Cell}, \emph{1}(5), 555-567. \url{https://doi.org/10.1016/j.stem.2007.08.014}

Hafkamp, H. C., Manni, J. J., Haesevoets, A., Voogd, A. C., Schepers, M., Bot, F. J., Hopman, A. H., Ramaekers, F. C., \& Speel, E. J. (2008). Marked differences in survival rate between smokers and nonsmokers with HPV 16-associated tonsillar carcinomas. \emph{Int J Cancer}, \emph{122}(12), 2656-2664. \url{https://doi.org/10.1002/ijc.23458}

Heinrich, P. C., Müller, M., \& Graeve, L. (2014). \emph{Löffler/Petrides Biochemie und Pathobiochemie}. Springer Berlin Heidelberg. \url{https://books.google.de/books?id=EUj_ugAACAAJ}

Herrmann, K., \& Niedobitek, G. (2003). Epstein-Barr virus-associated carcinomas: facts and fiction. \emph{J Pathol}, \emph{199}(2), 140-145. \url{https://doi.org/10.1002/path.1296}

Joos, S., Nettelbeck, D. M., Reil-Held, A., Engelmann, K., Moosmann, A., Eggert, A., Hiddemann, W., Krause, M., Peters, C., Schuler, M., Schulze-Osthoff, K., Serve, H., Wick, W., Puchta, J., \& Baumann, M. (2019). German Cancer Consortium (DKTK) - A national consortium for translational cancer research. \emph{Mol Oncol}, \emph{13}(3), 535-542. \url{https://doi.org/10.1002/1878-0261.12430}

Jütz, M., Linge, A., von Neubeck, C., Lohaus, F., Tinhofer, I., Budach, V., Gkika, E., Stuschke, M., Balermpas, P., Rödel, C., Avlar, M., Grosu, A. L., Abdollahi, A., Debus, J., Bayer, C., Belka, C., Pigorsch, S., Combs, S. E., Mönnich, D., . . . DKTK-ROG. (2015). Prognostisches Potential von CD44 als Tumorstammzellmarker für die kombinierte Radiochemotherapie des lokal fortgeschrittenen Kopf-Hals-Plattenepithelkarzinoms. Symposium Experimentelle Strahlentherapie und klinische Strahlenbiologie,

Klijanienko, J., el-Naggar, A., Ponzio-Prion, A., Marandas, P., Micheau, C., \& Caillaud, J. M. (1993). Basaloid squamous carcinoma of the head and neck. Immunohistochemical comparison with adenoid cystic carcinoma and squamous cell carcinoma. \emph{Arch Otolaryngol Head Neck Surg}, \emph{119}(8), 887-890. \url{https://doi.org/10.1001/archotol.1993.01880200093013}

Klussmann, J. P., Weissenborn, S. J., Wieland, U., Dries, V., Eckel, H. E., Pfister, H. J., \& Fuchs, P. G. (2003). Human papillomavirus-positive tonsillar carcinomas: a different tumor entity? \emph{Med Microbiol Immunol}, \emph{192}(3), 129-132. \url{https://doi.org/10.1007/s00430-002-0126-1}

Klussmann, J. P., Weissenborn, S. J., Wieland, U., Dries, V., Kolligs, J., Jungehuelsing, M., Eckel, H. E., Dienes, H. P., Pfister, H. J., \& Fuchs, P. G. (2001). Prevalence, distribution, and viral load of human papillomavirus 16 DNA in tonsillar carcinomas. \emph{Cancer}, \emph{92}(11), 2875-2884. \url{https://doi.org/10.1002/1097-0142(20011201)92:11}\textless2875::aid-cncr10130\textgreater3.0.co;2-7

Kononen, J., Bubendorf, L., Kallioniemi, A., Barlund, M., Schraml, P., Leighton, S., Torhorst, J., Mihatsch, M. J., Sauter, G., \& Kallioniemi, O. P. (1998). Tissue microarrays for high-throughput molecular profiling of tumor specimens. \emph{Nat Med}, \emph{4}(7), 844-847. \url{https://doi.org/10.1038/nm0798-844}

Krause, M., Yaromina, A., Eicheler, W., Koch, U., \& Baumann, M. (2011). Cancer stem cells: targets and potential biomarkers for radiotherapy. \emph{Clin Cancer Res}, \emph{17}(23), 7224-7229. \url{https://doi.org/10.1158/1078-0432.CCR-10-2639}

Lenarz, T., \& Boenninghaus, H. G. (2012). \emph{Hals-Nasen-Ohren-Heilkunde}. Springer Berlin Heidelberg. \url{https://books.google.ne/books?id=RLGr7gRB0kUC}

Lindel, K., Beer, K. T., Laissue, J., Greiner, R. H., \& Aebersold, D. M. (2001). Human papillomavirus positive squamous cell carcinoma of the oropharynx: a radiosensitive subgroup of head and neck carcinoma. \emph{Cancer}, \emph{92}(4), 805-813. \url{https://doi.org/10.1002/1097-0142(20010815)92:4}\textless805::aid-cncr1386\textgreater3.0.co;2-9

Lindquist, D., Romanitan, M., Hammarstedt, L., Näsman, A., Dahlstrand, H., Lindholm, J., Onelöv, L., Ramqvist, T., Ye, W., Munck-Wikland, E., \& Dalianis, T. (2007). Human papillomavirus is a favourable prognostic factor in tonsillar cancer and its oncogenic role is supported by the expression of E6 and E7. \emph{Mol Oncol}, \emph{1}(3), 350-355. \url{https://doi.org/10.1016/j.molonc.2007.08.005}

Lingen, M. W. (2000). Lucas' pathology of tumors of the oral tissues. \emph{Arch Pathol Lab Med}, \emph{124}(3), 475. \url{https://www.ncbi.nlm.nih.gov/pubmed/10705417}

Lohaus, F., Linge, A., Tinhofer, I., Budach, V., Gkika, E., Stuschke, M., Balermpas, P., Rodel, C., Avlar, M., Grosu, A. L., Abdollahi, A., Debus, J., Bayer, C., Belka, C., Pigorsch, S., Combs, S. E., Monnich, D., Zips, D., von Neubeck, C., . . . Dktk, R. O. G. (2014). HPV16 DNA status is a strong prognosticator of loco-regional control after postoperative radiochemotherapy of locally advanced oropharyngeal carcinoma: results from a multicentre explorative study of the German Cancer Consortium Radiation Oncology Group (DKTK-ROG). \emph{Radiother Oncol}, \emph{113}(3), 317-323. \url{https://doi.org/10.1016/j.radonc.2014.11.011}

Mack, B., \& Gires, O. (2008). CD44s and CD44v6 expression in head and neck epithelia. \emph{PLoS One}, \emph{3}(10), e3360. \url{https://doi.org/10.1371/journal.pone.0003360}

MacMillan, C., Kapadia, S. B., Finkelstein, S. D., Nalesnik, M. A., \& Barnes, L. (1996). Lymphoepithelial carcinoma of the larynx and hypopharynx: study of eight cases with relationship to Epstein-Barr virus and p53 gene alterations, and review of the literature. \emph{Hum Pathol}, \emph{27}(11), 1172-1179. \url{https://doi.org/10.1016/s0046-8177(96)90311-1}

Mashberg, A., \& Samit, A. (1995). Early diagnosis of asymptomatic oral and oropharyngeal squamous cancers. \emph{CA Cancer J Clin}, \emph{45}(6), 328-351. \url{https://doi.org/10.3322/canjclin.45.6.328}

Nocito, A., Kononen, J., Kallioniemi, O. P., \& Sauter, G. (2001). Tissue microarrays (TMAs) for high-throughput molecular pathology research. \emph{Int J Cancer}, \emph{94}(1), 1-5. \url{https://doi.org/10.1002/ijc.1385}

Odell, E. W., Farthing, P. M., High, A., Potts, J., Soames, J., Thakker, N., Toner, M., \& Williams, H. K. (2004). British Society for Oral and Maxillofacial Pathology, UK: minimum curriculum in oral pathology. \emph{Eur J Dent Educ}, \emph{8}(4), 177-184. \url{https://doi.org/10.1111/j.1600-0579.2004.00350.x}

Paz, I. B., Cook, N., Odom-Maryon, T., Xie, Y., \& Wilczynski, S. P. (1997). Human papillomavirus (HPV) in head and neck cancer. An association of HPV 16 with squamous cell carcinoma of Waldeyer's tonsillar ring. \emph{Cancer}, \emph{79}(3), 595-604. \url{https://doi.org/10.1002/(sici)1097-0142(19970201)79:3}\textless595::aid-cncr24\textgreater3.0.co;2-y

Prince, M. E., Sivanandan, R., Kaczorowski, A., Wolf, G. T., Kaplan, M. J., Dalerba, P., Weissman, I. L., Clarke, M. F., \& Ailles, L. E. (2007). Identification of a subpopulation of cells with cancer stem cell properties in head and neck squamous cell carcinoma. \emph{Proc Natl Acad Sci U S A}, \emph{104}(3), 973-978. \url{https://doi.org/10.1073/pnas.0610117104}

Raslan, W. F., Barnes, L., Krause, J. R., Contis, L., Killeen, R., \& Kapadia, S. B. (1994). Basaloid squamous cell carcinoma of the head and neck: a clinicopathologic and flow cytometric study of 10 new cases with review of the English literature. \emph{Am J Otolaryngol}, \emph{15}(3), 204-211. \url{https://doi.org/10.1016/0196-0709(94)90006-x}

Ritchie, J. M., Smith, E. M., Summersgill, K. F., Hoffman, H. T., Wang, D., Klussmann, J. P., Turek, L. P., \& Haugen, T. H. (2003). Human papillomavirus infection as a prognostic factor in carcinomas of the oral cavity and oropharynx. \emph{Int J Cancer}, \emph{104}(3), 336-344. \url{https://doi.org/10.1002/ijc.10960}

RKI. (2021). \emph{Krebs in Deutschland für 2017/2018 13. Auflage Berlin}.

Sabatini, M. E., \& Chiocca, S. (2020). Human papillomavirus as a driver of head and neck cancers. \emph{British Journal of Cancer}, \emph{122}(3), 306-314. \url{https://doi.org/10.1038/s41416-019-0602-7}

Schmincke, A. (1921). \emph{Beitrage zur pathologischen Anatomie und allgemeinen Pathologie}. Gustav Fischer Verlag. \url{https://books.google.de/books?id=F3pMAQAAMAAJ}

Shi, W., Kato, H., Perez-Ordonez, B., Pintilie, M., Huang, S., Hui, A., O'Sullivan, B., Waldron, J., Cummings, B., Kim, J., Ringash, J., Dawson, L. A., Gullane, P., Siu, L., Gillison, M., \& Liu, F. F. (2009). Comparative prognostic value of HPV16 E6 mRNA compared with in situ hybridization for human oropharyngeal squamous carcinoma. \emph{J Clin Oncol}, \emph{27}(36), 6213-6221. \url{https://doi.org/10.1200/JCO.2009.23.1670}

Smith, E. M., Ritchie, J. M., Summersgill, K. F., Klussmann, J. P., Lee, J. H., Wang, D., Haugen, T. H., \& Turek, L. P. (2004). Age, sexual behavior and human papillomavirus infection in oral cavity and oropharyngeal cancers. \emph{Int J Cancer}, \emph{108}(5), 766-772. \url{https://doi.org/10.1002/ijc.11633}

Snijders, P. J., Cromme, F. V., van den Brule, A. J., Schrijnemakers, H. F., Snow, G. B., Meijer, C. J., \& Walboomers, J. M. (1992). Prevalence and expression of human papillomavirus in tonsillar carcinomas, indicating a possible viral etiology. \emph{Int J Cancer}, \emph{51}(6), 845-850. \url{https://doi.org/10.1002/ijc.2910510602}

Sturgis, E. M., \& Cinciripini, P. M. (2007). Trends in head and neck cancer incidence in relation to smoking prevalence: an emerging epidemic of human papillomavirus-associated cancers? \emph{Cancer}, \emph{110}(7), 1429-1435. \url{https://doi.org/10.1002/cncr.22963}

Syrjanen, K., Syrjanen, S., \& Pyrhonen, S. (1982). Human papilloma virus (HPV) antigens in lesions of laryngeal squamous cell carcinomas. \emph{ORL J Otorhinolaryngol Relat Spec}, \emph{44}(6), 323-334. \url{https://doi.org/10.1159/000275612}

Thompson, L. (2006). World Health Organization classification of tumours: pathology and genetics of head and neck tumours. \emph{Ear Nose Throat J}, \emph{85}(2), 74. \url{https://www.ncbi.nlm.nih.gov/pubmed/16579185}

Thompson, L. D., Wieneke, J. A., Miettinen, M., \& Heffner, D. K. (2002). Spindle cell (sarcomatoid) carcinomas of the larynx: a clinicopathologic study of 187 cases. \emph{Am J Surg Pathol}, \emph{26}(2), 153-170. \url{https://doi.org/10.1097/00000478-200202000-00002}

Thompson, L. D. R., \& Bishop, J. A. (2017). \emph{Head and Neck Pathology E-Book: A Volume in the Series: Foundations in Diagnostic Pathology}. Elsevier Health Sciences. \url{https://books.google.de/books?id=zNZCDwAAQBAJ}

Thurnher, D., Grasl, M., Erovic, B. M., \& Lercher, P. (2010). \emph{HNO-Heilkunde: Ein symptomorientiertes Lehrbuch}. Springer Vienna. \url{https://books.google.de/books?id=RnJovwEACAAJ}

Vokes, E. E., Weichselbaum, R. R., Lippman, S. M., \& Hong, W. K. (1993). Head and neck cancer. \emph{N Engl J Med}, \emph{328}(3), 184-194. \url{https://doi.org/10.1056/NEJM199301213280306}

Weinberger, P. M., Yu, Z., Haffty, B. G., Kowalski, D., Harigopal, M., Brandsma, J., Sasaki, C., Joe, J., Camp, R. L., Rimm, D. L., \& Psyrri, A. (2006). Molecular classification identifies a subset of human papillomavirus--associated oropharyngeal cancers with favorable prognosis. \emph{J Clin Oncol}, \emph{24}(5), 736-747. \url{https://doi.org/10.1200/JCO.2004.00.3335}

Wicha, M. S., Liu, S., \& Dontu, G. (2006). Cancer stem cells: an old idea--a paradigm shift. \emph{Cancer Res}, \emph{66}(4), 1883-1890; discussion 1895-1886. \url{https://doi.org/10.1158/0008-5472.CAN-05-3153}

Wittekind, C. (2013). \emph{TNM-Supplement: Erlauterungen Zur Einheitlichen Anwendung}. Wiley-VCH. \url{https://books.google.de/books?id=8MGpBAAAQBAJ}

Zhang, Q., Shi, S., Yen, Y., Brown, J., Ta, J. Q., \& Le, A. D. (2010). A subpopulation of CD133(+) cancer stem-like cells characterized in human oral squamous cell carcinoma confer resistance to chemotherapy. \emph{Cancer Lett}, \emph{289}(2), 151-160. \url{https://doi.org/10.1016/j.canlet.2009.08.010}

\begin{center}\rule{0.5\linewidth}{0.5pt}\end{center}

\hypertarget{references}{%
\subsection{References}\label{references}}

\hypertarget{refs}{}
\begin{CSLReferences}{0}{0}
\leavevmode\vadjust pre{\hypertarget{ref-IhliSZDo}{}}%
\CSLLeftMargin{1. }%
\CSLRightInline{\textbf{Sci-Hub provides access to nearly all scholarly literature}
\CSLBlock{Daniel S Himmelstein, Ariel Rodriguez Romero, Jacob G Levernier, Thomas Anthony Munro, Stephen Reid McLaughlin, Bastian Greshake Tzovaras, Casey S Greene} \emph{eLife} (2018-03-01) \url{https://doi.org/ckcj}
\CSLBlock{DOI: \href{https://doi.org/10.7554/elife.32822}{10.7554/elife.32822} · PMID: \href{https://www.ncbi.nlm.nih.gov/pubmed/29424689}{29424689} · PMCID: \href{https://www.ncbi.nlm.nih.gov/pmc/articles/PMC5832410}{PMC5832410}}}

\leavevmode\vadjust pre{\hypertarget{ref-mSMVRkoc}{}}%
\CSLLeftMargin{2. }%
\CSLRightInline{\textbf{Reproducibility of computational workflows is automated using continuous analysis}
\CSLBlock{Brett K Beaulieu-Jones, Casey S Greene} \emph{Nature biotechnology} (2017-04) \url{https://www.ncbi.nlm.nih.gov/pmc/articles/PMC6103790/}
\CSLBlock{DOI: \href{https://doi.org/10.1038/nbt.3780}{10.1038/nbt.3780} · PMID: \href{https://www.ncbi.nlm.nih.gov/pubmed/28288103}{28288103} · PMCID: \href{https://www.ncbi.nlm.nih.gov/pmc/articles/PMC6103790}{PMC6103790}}}

\leavevmode\vadjust pre{\hypertarget{ref-126Wi5Us4}{}}%
\CSLLeftMargin{3. }%
\CSLRightInline{\textbf{Bitcoin for the biological literature.}
\CSLBlock{Douglas Heaven} \emph{Nature} (2019-02) \url{https://www.ncbi.nlm.nih.gov/pubmed/30718888}
\CSLBlock{DOI: \href{https://doi.org/10.1038/d41586-019-00447-9}{10.1038/d41586-019-00447-9} · PMID: \href{https://www.ncbi.nlm.nih.gov/pubmed/30718888}{30718888}}}

\leavevmode\vadjust pre{\hypertarget{ref-QhC8yJ7V}{}}%
\CSLLeftMargin{4. }%
\CSLRightInline{\textbf{Plan S: Accelerating the transition to full and immediate Open Access to scientific publications}
\CSLBlock{cOAlition S} (2018-09-04) \url{https://www.wikidata.org/wiki/Q56458321}}

\leavevmode\vadjust pre{\hypertarget{ref-zBPP9YKu}{}}%
\CSLLeftMargin{5. }%
\CSLRightInline{\textbf{Open access}
\CSLBlock{Peter Suber} \emph{MIT Press} (2012)}

\leavevmode\vadjust pre{\hypertarget{ref-1GGGHdsew}{}}%
\CSLLeftMargin{6. }%
\CSLRightInline{\textbf{Open collaborative writing with Manubot}
\CSLBlock{Daniel S Himmelstein, Vincent Rubinetti, David R Slochower, Dongbo Hu, Venkat S Malladi, Casey S Greene, Anthony Gitter} \emph{Manubot} (2020-05-25) \url{https://greenelab.github.io/meta-review/}}

\leavevmode\vadjust pre{\hypertarget{ref-PZMP42Ak}{}}%
\CSLLeftMargin{7. }%
\CSLRightInline{\textbf{Opportunities and obstacles for deep learning in biology and medicine}
\CSLBlock{Travers Ching, Daniel S Himmelstein, Brett K Beaulieu-Jones, Alexandr A Kalinin, Brian T Do, Gregory P Way, Enrico Ferrero, Paul-Michael Agapow, Michael Zietz, Michael M Hoffman, \ldots{} Casey S Greene} \emph{Journal of The Royal Society Interface} (2018-04) \url{https://doi.org/gddkhn}
\CSLBlock{DOI: \href{https://doi.org/10.1098/rsif.2017.0387}{10.1098/rsif.2017.0387} · PMID: \href{https://www.ncbi.nlm.nih.gov/pubmed/29618526}{29618526} · PMCID: \href{https://www.ncbi.nlm.nih.gov/pmc/articles/PMC5938574}{PMC5938574}}}

\leavevmode\vadjust pre{\hypertarget{ref-YuJbg3zO}{}}%
\CSLLeftMargin{8. }%
\CSLRightInline{\textbf{Open collaborative writing with Manubot}
\CSLBlock{Daniel S Himmelstein, Vincent Rubinetti, David R Slochower, Dongbo Hu, Venkat S Malladi, Casey S Greene, Anthony Gitter} \emph{PLOS Computational Biology} (2019-06-24) \url{https://doi.org/c7np}
\CSLBlock{DOI: \href{https://doi.org/10.1371/journal.pcbi.1007128}{10.1371/journal.pcbi.1007128} · PMID: \href{https://www.ncbi.nlm.nih.gov/pubmed/31233491}{31233491} · PMCID: \href{https://www.ncbi.nlm.nih.gov/pmc/articles/PMC6611653}{PMC6611653}}}

\end{CSLReferences}
